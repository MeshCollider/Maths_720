\documentclass[a4paper,10pt]{article}
\usepackage{amsmath,amsfonts,amsthm,amssymb,fullpage,enumerate}
\usepackage{pgf,tikz}
\usepackage{graphicx}
\usepackage{setspace}
\usepackage{epstopdf}
\usepackage{MnSymbol}
\newcommand{\CC}{\mathbb{C}}
\newcommand{\RR}{\mathbb{R}}
\newcommand{\NN}{\mathbb{N}}
\newcommand{\QQ}{\mathbb{Q}}
\newcommand{\ZZ}{\mathbb{Z}}
\newcommand{\EE}{\mathbb{E}}
\newcommand{\FF}{\mathbb{F}}
\newcommand{\GG}{\mathcal{G}}
\newcommand{\PP}{\mathbb{P}}
\newcommand{\BB}{\mathcal{B}}
\newcommand{\dd}{\,\mathrm{d}}
\newcommand{\II}{\mathbb{I}}
\newcommand{\ul}{\underline}
\newcommand{\zero}{\{0\}}			
\newcommand{\va}{\mathbf{a}}
\newcommand{\vx}{\mathbf{x}}
\providecommand{\norm}[1]{\|\cdot\|_{#1}}
\providecommand{\normm}[2]{\left\|{#1}\right\|_{#2}}
\providecommand{\ip}[1]{\bds\left( #1 \right)\eds}
\DeclareMathOperator{\interior}{int}
\DeclareMathOperator{\exterior}{ext}
\newtheorem{thm}{Theorem}
\newtheorem{Def}[thm]{Definition}
\newtheorem{Cor}[thm]{Corollary}
\newtheorem{cl}[thm]{Claim}
\newtheorem{prop}[thm]{Proposition}
\newtheorem{eg}[thm]{Example}
\newtheorem{Ex}[thm]{Exercise}
\newtheorem{Lem}[thm]{Lemma}
\newtheorem{note}[thm]{Note}
\newtheorem{rem}[thm]{Remark}
\newcommand{\bds}{\begin{displaystyle}}
\newcommand{\eds}{\end{displaystyle}}
\newcommand{\bpm}{\begin{pmatrix}}
\newcommand{\epm}{\end{pmatrix}}
\newcommand{\bvm}{\begin{vmatrix}}
\newcommand{\evm}{\end{vmatrix}}
\newcommand{\conj}{\overline}
\newcommand{\inv}{^{-1}}
\newcommand{\Ito}{It$\hat{\text{o}}$} 
\newcommand{\cadlag}{\textit{c\`{a}dl\`{a}g}}
\newcommand{\Cadlag}{\textit{C\`{a}dl\`{a}g}}

\setlength{\parindent}{0pt}

\begin{document}

\title{Maths 720 Notes}
\author{Louis Christie \\ $6652368$}
\maketitle


\onehalfspacing



\section{Monday $26^{th}$ February}
$D_3 \simeq S_3 \simeq GL(2, \FF_2)$
\begin{Def}
V is an elementary $p$-group if $|g| \big|\, p$ for all $g \in V$
\end{Def}
\begin{prop}
If $|G| = p^2$ for some prime $p$ then $Z(G) = G$
\end{prop}

Note when $x = (1234)$, $y = (14)(23)$,
\[ D_4 = \langle (1234), (14)(23)\rangle = \{ x, x^2, x^3, x^4 = e = y^2, y, xy, yx, x^2y = yx^2 \} \]
\begin{cl}
$D_4$ has 5 elements of order 2
\end{cl}
\begin{cl}
$Z(G) = \langle (13)(24) \rangle$
\end{cl}
Let $Q_8 \leq S_8 = \langle (1625)(3847),(1423)(5768)\rangle$ then
\[ Z(Q_8) = \langle (12)(34)(56)(78)\rangle \qquad |Z(Q_8)| = 2 \]
\begin{cl}
$Q_8$ has a unique element of order 2
\end{cl}
\begin{Cor}
$Q_8 \not\simeq D_4$
\end{Cor}
Another way of creating $Q_8$ is as a subgroup of $GL(2,\CC)$. Set:
\[ x = \begin{pmatrix}
 i & 0 \\ 0 & -i
 \end{pmatrix} \qquad y = \begin{pmatrix}
 0 & 1 \\ -1 & 0
 \end{pmatrix} \]
Then set $Q_8 = \langle x,y\rangle \leq GL(2,\CC)$.
\begin{Def}
Let
\begin{align*}
D_n &= \langle (12\dots n),(1,n)(2,n-1)\dots(\lfloor n / 2 \rfloor, \lfloor n / 2 \rfloor +1) \rangle \\
S_n  &= \langle (12),(12\dots n) \rangle \\
A_n &= \langle (123) \{ n-1 \text{cycle or } n-2 \text{cycle} \}
\end{align*}
\end{Def}
Note $|S_n| = n!$, $|A_n| = n! / 2$.
\begin{eg}
It is clear that:
\[A_5= \langle (123),(345) \rangle \qquad |A_5| = 60 \]
and 
\[A_6 = \langle (123), (23456) \rangle \qquad |A_6| = 360 \]
\end{eg}
\begin{Ex}
Find $A_4$ and $S_4$
\end{Ex}
\begin{Def}
Set $V = \FF_q^d$: a $d$-dimensional vector space over $\FF_q$. If $\FF$ is finite, then $|\FF| = p^e$ for some prime $p$ and $e \in \ZZ_+$. Now let $GL(V)$ be the group of linear transformations of $V$. 
\end{Def}
\begin{rem}
Note $GL(V) \simeq GL(d, \FF_q)$.
\end{rem}
\begin{Ex}
Find $|GL(d,\FF_q)$ (\textit{Hint: $GL(2,2) = 6$, $GL(2,3) = 48$, $GL(3,2) = 68$. $GL(2,2) = (q^d - 1) \times ?$})
\end{Ex}

\newpage
\section{Tuesday $27^{th}$ February}

\begin{Def}
For $H \leq G$ and $x \in G$ we define the cosets of $H$ as:
\begin{align*}
Hx &= \{hx : h \in H \} \\
xH &= \{xh : h \in H \}
\end{align*}

\end{Def}
\begin{prop}
For any $H \leq G$, we have:
\[ G = \bigcup_{i=0}^k Hx_i = \bigcup_{j=0}^k x_jH \]
For some $k \in \NN$ and $(x_i), (x_j)$.
\end{prop}
\begin{Def}
We call such $k$ the index of $H$ in $G$, and write $k = |G : H|$. 
\end{Def}
\begin{thm}
If $G$ is a finite group and $H \leq G$ then $|H| \big| |G|$.
\end{thm}
\begin{cl}
If $H \leq G$ and $x \in G$, then $|H| = |Hx| = |xH|$.
\end{cl}
\begin{Def}
We call $H \leq G$ normal in $G$ if $xH = Hx$ for all $x \in G$. We write $H \triangleleft G$. 
\end{Def}
\begin{rem}
This is equivalent to $x^{-1}Hx = H$ and $H^x = H$ for all $x \in G$
\end{rem}
\begin{Def}
If $H \triangleleft G$ then we define $Q = G / H = \{Hx : x \in G\}$, i.e. the set of (right) cosets of $H$.
\end{Def}
\begin{eg}
Set $G = S_3 = \langle a,b\rangle$, with $a = (123)$ and $b = (12)$, and $H = \langle b \rangle$. Then $H^a = \langle (13) \rangle$, so $H$ is not normal in $G$. Set $H' = \langle a \rangle$. Then $H' \triangleleft G$.
\end{eg}
\begin{cl}
If $H \leq G$ and $|G : H| = 2$ then $H \triangleleft G$.
\end{cl}

\begin{Def}
Given the set $Q = G / H$, define a multiplication on this set $\cdot : Q \rightarrow Q$ by:
\[ (Hx) \cdot (Hy) = H(xy) \]
\end{Def}
\begin{prop}
$(G / H, \cdot)$ is a group
\end{prop}
\begin{Def}
For groups $G,H$, We call a function $\phi : G \rightarrow H$ a homomorphism if
\[ \phi(xy) = \phi(x) \phi(y) \]
We call a bijective homomorphism an isomorphism. If $\phi : G \rightarrow G$ is isomorphic then we call it an automorphism.
\end{Def}
\begin{eg}
Take $G = GL(d, \FF)$, and $H = \FF$. Then $\phi : G \rightarrow H$ given by $g \mapsto det(g)$ is a homomorphism.
\end{eg}
\begin{Def}
For a homomorphism $\phi : G \rightarrow H$, we define $\ker \phi = \{ g \in G : \phi(g) = 1 \}$.
\end{Def}
\begin{eg}
Again take $\phi : GL(d, \FF) \rightarrow \FF$ given by $g \mapsto det(g)$. Then 
\[ \ker \phi = \{ x \in G : \det(x) = 1 \} = SL(d,\FF) \triangleleft GL(d, \FF) \] 
We call $SL(d,\FF)$ the special linear group, and $|SL(n,\FF_q)| = |GL(n,\FF_q) | / (q - 1)$
\end{eg}
\begin{Def}
For $H \triangleleft G$, define $\phi : G \rightarrow G / H$ by $g \mapsto Hg$.
\end{Def}
\begin{rem}
Observe $\phi$ is a homomorphism. In fact it is the canonical Homomorphism from $G$ to $G / H$. 
\end{rem}
\begin{cl}
For any homomorphism $\phi : G \rightarrow H$, $\ker \phi \triangleleft G$.
\end{cl}
\begin{thm}[First Isomorphism Theorem]
Let $K = \ker \phi$, for some surjective homomorphism $\phi : G \rightarrow H$. Then $H \simeq G / \ker \phi$.
\end{thm}
\begin{thm}[Correspondence Theorem]
Let $G$ be a group and $N \triangleleft G$. Then:
\begin{enumerate}
\item If $N \leq A \leq G$ then $A / N \leq G / N$. If $N \subseteq A \triangleleft G$, then $A / N \triangleleft G \ N$.
\item Every $A \leq G / N$ has the form $B / N$ for some suitable $B \leq G$. Every $A \triangleleft G / N$ has the form $B / N$ for some suitable $B \triangleleft G$.
\end{enumerate}
\end{thm}
(TODO: Add figure)
\begin{proof}
For part (1), we claim that if $N \triangleleft G$ then $N \triangleleft A$, in which case $A / N$ exists and $A / N \subseteq G / N$. Also, $A / N$ is non-empty ( as $N \in A / N$ ), so take $Na, Nb \in A/N$. Then
\[ (Na)(Nb)^{-1} = N(ab^{-1}) \in A / N \]
Thus $A / N \leq G / N$. If $A \triangleleft G$, then $g^{-1} a g \in A$ for all $g \in G$, so:
\[ (Ng)^{-1} (Na) (Ng) = N(g^{-1}ag) \in A / N \]
and thus $A / N \triangleleft G / N$.  \\
\\
For part (2), it reduces to the First Isomorphism Theorem. 
\end{proof}

\begin{Def}
Take $H,K \triangleleft G$, and define
\[ HK = \{ hk : h\in H, k\in K \} \]
\end{Def}
\begin{Lem}
$HK \leq G$ if, and only if, $HK = KH$. 
\end{Lem}
\begin{proof}
(TODO : add proof)
\end{proof}
\begin{Cor}
If $N \triangleleft G$ and $H \leq G$, then $NH \leq G$. 
\end{Cor}
\begin{Cor}
If $N \triangleleft G$ and $H \leq G$, then $\langle N, H \rangle = NH$. This is because clearly $NH \leq \langle N, H \rangle$, but also $NH \leq G$, so as $\langle N, H \rangle = NH$ is the smallest subgroup containing $NH$, they must be equal.
\end{Cor}
\begin{eg}
Take $G = D_4 = \langle a , b \rangle$ where $a = (1234), b =(14)(23)$. Let $N = \langle a^2 \rangle \triangleleft G$. Then $| G / N | = 4$. Also, $G / N \simeq C_2 \times C_2$.
(TODO : add diagram)
\end{eg}


\newpage
\section{Thursday $1^{st}$ March}
\begin{Def}
We call $Z(G) = \{ y \in G : xy = yx \, \forall x \in G \}$ the center of $G$.
\end{Def}
\begin{Lem}
$Z(G) \triangleleft G$.
\end{Lem}
\begin{rem}
Observe that $Z(G)$ is abelian and $G$ is abelian if and only if $Z(G) = G$.
\end{rem}
\begin{eg}
$Z(S_3) = 1$. In fact, $Z(S_n) = 1$ for every $n \geq 3$. Also, for $G = GL(2, \RR)$:
\[ A \in Z(G) \implies A = a \begin{pmatrix} 1 & 0 \\ 0 & 1 \end{pmatrix}, \text{i.e., } Z(G) = \left\{ \begin{pmatrix} a & 0 \\ 0 & a \end{pmatrix} : x \in \RR \right\}  \]
$Z(D_4) = \langle (13)(24) \rangle$. 
\end{eg}
\begin{Def}
We define the centraliser of $g \in G$ as 
\[ C_G(g) = \{ x \in G : xg = gx \} \]
\end{Def}
\begin{rem}
Note that $C_G(g) \neq \varnothing$ and $C_G(g) \leq G$. 
\end{rem}
\begin{eg}
$Z(D_4) = \langle (13)(24) \rangle$. Then
\[ C_{D_4} \big( (12)(34) \big) = \langle (13)(24), (12)(34) \rangle \]
\end{eg}
\begin{Ex}
Find $Z(D_n)$ (note that $|Z(D_n)| = 2 - (n \% 2)$.
\end{Ex}
\begin{prop}
If $\phi: G_1 \rightarrow G_2$ is an isomorphism, then
\[ \phi(Z(G_1)) = Z(G_2). \]
If $\phi$ is an automorphism on $G$, then $\phi(Z(G)) = Z(G)$. We say $Z$ is fixed under automorphisms. 
\end{prop}

\begin{Def}
We say $C$ is a characteristic of $G$ if $C \phi = C$ for every automorphism $\phi$.
(TODO : Check with Eamonn)
We say $C \leq G$ is a characteristic subgroup of $G$ if $\phi(C) \leq C$ for every $\phi \in \text{Aut}(G)$. 
\end{Def}
\begin{Lem}
The set of automorphisms of $G$, paired with composition, forms a group. 
\end{Lem}
\begin{Def}
Let $g \in G$. The define the inner automorphism of $G$ generated by $g$ as:
\[ \theta_g(x) = g^{-1} x g = x^g \]
Observe that $\theta_g$ does indeed form an automorphism of $G$. Also define $\text{Inn} (G) = \{\theta_g : g \in G \}$. 
\end{Def}
\begin{Lem}
$\text{Inn}(G) \triangleleft \text{Aut}(G)$
\end{Lem}
\begin{proof}
Exercise.
\end{proof}
\begin{rem}
Observe that if $G$ is abelian then $\text{Inn}(G) = 1$. 
\end{rem}
\begin{rem}
Characteristic subgroups are fixed under automorphisms, normal subgroups are fixed under inner automorphisms
\end{rem}
\begin{Def}
Take 
\[ \text{Out} (G) = \frac{\text{Aut}(G)}{\text{Inn}(G)} \]
Note that often $\theta \in \text{Aut}(G) \setminus \text{Inn}(G)$ is called an outer automorphism, but these are distinct. 
\end{Def}
\begin{Ex}
Show $\text{Inn}(G) \simeq G / Z(G)$. (Hint: $1^{st}$ Isomorphism theorem. Define $\psi : G \rightarrow \text{Inn}(G)$ with $\ker \psi = Z(G)$.)
\end{Ex}
\begin{prop}
For any $\phi \in \text{Aut}(G)$, if $G = \langle g \rangle$ then $G = \langle \phi(g) \rangle$.
\end{prop}

\begin{eg}
Take $(\ZZ, +) (= \langle x \rangle, x \in \{1,-1\})$. Thus 
\[ \text{Aut} \big( (\ZZ, +) \big) = \{ \phi_a : x \mapsto x, \phi_b : x \mapsto -x\} \simeq \ZZ_2 \]
\end{eg}

\begin{Lem}
Let $G = \langle x \rangle \simeq \ZZ_n$. Then take $0 \leq m < n$. Define $\iota_m : G \rightarrow G$ by $x \mapsto x^m$. Then
\[ \text{Aut}(G) = \{ \iota_m : \gcd(m,n) = 1 \} \]
\end{Lem}

\begin{Cor}
If p is prime then $\text{Aut}(\ZZ_p) = \ZZ_{p-1}$. 
\end{Cor}

\begin{Def}
We call $G$ an elementary abelian group if $G \simeq (C_p )^d$. 
\end{Def}

\begin{rem}
Note that every element of an elementary group has order $1$ or $p$.
\end{rem}

\begin{thm}
Let $G \simeq (C_p)^d$. Then $\text{Aut}(G) \simeq GL(d,p)$. 
\end{thm}

\begin{rem}
Characteristic subgroups are fixed under automorphisms, normal subgroups are fixed under inner automorphisms
\end{rem}

\begin{Def}
Let $H \leq G$. Then define the normaliser of $H$ in $G$ as:
\[ N_G(H) = \{ g \in G : g^{-1}Hg = H \} \]
\end{Def}
\begin{Lem}
$N_G(H) \leq G$. 
\end{Lem}
\begin{rem}
$H \triangleleft G$ if and only if $N_G(H) = G$. 
\end{rem}
\begin{eg}
Let $G = S_3$. Then:
\begin{align*}
H_1 &= \langle (12)\rangle \implies N_G(H_1) = H_1 \\
H_2 &= \langle (123) \rangle \implies N_G(H_2) = H_2 
\end{align*}

\end{eg}

\begin{Def}
Let $x,y \in G$. Then define:
\[ [x,y] = x^{-1} y^{-1} xy \]
We call this the commutator of $x,y$. 
\end{Def}
\begin{eg}
Which $c \in S_3$ are commutators? We know All commutators must have even order, so $x \in A_3 = \{ 1, (123), (132) \}$. But we also have:
\begin{align*}
[(12), (12)] &= 1 \\
[(12), (23)] &= (123) \\
[(12), (13)] &= (132)
\end{align*}
So all elements of $A_3$ are commutators of $S_3$
\end{eg}
\begin{Def}
The derived group $G' \leq G$ is defined as:
\[ G' = \langle [x,y] : x,y \in G \rangle \]
I.e., the group generated by all commutators. 
\end{Def}
\begin{eg}
$S_3 ' = A_3$, as shown. In general $S_n ' = A_n$ for all $n \geq 3$. Prove as exercise.
\end{eg}

\begin{Lem}
$G ' \triangleleft G$.
\end{Lem}

\begin{Lem}
Both $Z(G)$ and $G'$ are characteristic subgroups of $G$. And are both therefore also normal. 
\end{Lem}


\begin{eg}
Take $A_4 = G$. $G' = \langle (13)(24), (12)(34) \rangle = \ZZ_2 \times \ZZ_2$. Thus $G' \triangleleft G$.
\end{eg}

\begin{rem}
There exists a group $G$ with order 96. It has 29 commutators, but $|G'| = 32$. This is the smallest group where $\{commutators\} \neq G'$.
\end{rem}
\begin{rem}
If $G$ is a finite simple group then every $g \in G' = G$ is a commutator.
\end{rem}


\newpage
\section{Monday $5^{th}$ March}

\begin{Lem}
Take $N \triangleleft G$, then 
\[ G / N \text{is abelian if and only if } G' \leq N \] 
\end{Lem}

\begin{proof}
$G/N$ is abelian iff $Nx \cdot Ny = Ny \cdot Nx$ for all $x,y \in G$. Which is equivalent to $N = Nx^{-1}y^{-1}xy$, which implies $[x,y] \in N$. $N$ all commutators, and hence $N \geq G'$. The converse follows exactly in reverse.
\end{proof}
\begin{Def}
Take homomorphism $\phi : G \rightarrow H$. Then if $U \leq G$,
\[ \phi(U) = \{ \phi(u) : u \in U \} \]
if $V \leq H$, then let
\[ \phi^{-1} (V) = \{ u \in G : \phi(u) \in V \} \]
\end{Def}

\begin{rem}
Whilst $\phi^{-1}$ can be defined, there need not be a map $\phi^{-1}$
\end{rem}

\begin{Lem}
Take homomorphism $\phi : G \rightarrow H$.
\begin{enumerate}
\item  If $U \leq G$ Then $\phi(U) \leq H$
\item If $V \leq H$ then $\phi^{-1}(V) \leq G$ that contains $\ker \phi$. 
\end{enumerate}

\end{Lem}

\begin{eg}
Take $N \triangleleft G$ and $\phi: G \rightarrow G / N$ the canonical homomorphism determined by $N$: $x \mapsto Nx$. Let $H \leq G$. Then what is $\phi(H)$?.
(TODO: Add diagram 1)
Observe $NH \leq G$, and $N \triangleleft NH$. Thus $NH / N \leq G / N$. Now
\[ NH / N = \{ Nnh : n \in N, h \in H \} = \{ Nh : h \in H \} = \phi(H) \]
\end{eg}

\begin{thm}[Second Isomorphism Theorem]
Take $N \triangleleft G$ and $H \leq G$. Then $N \cap H \triangleleft H$ and 
\[ H / (N \cap H) \simeq NH / N \]
\end{thm}

\begin{proof}
Define $\phi : G \rightarrow G / N$ by $g \mapsto Ng$. Then let $\psi$ be the restriction of $\phi$ to $H \leq G$. Then $\psi$ is a homomorphism from $H \rightarrow \psi(H) = NH / N$. Note that it is surjective. Now 
\[ \ker \psi = H \cap \ker \phi = H \cap N \]
But also $\ker \psi \triangleleft H$ and so $H \cap N \triangleleft H$. By the first isomorphism theorem:
\[ H / \ker \psi = H / (H \cap N) \simeq NH / N \]
\end{proof}

\begin{eg}
Take $G = S_4 = \langle (1234), (12) \rangle$, take $N = \langle (12)(34), (13)(24) \rangle$, and $H = \langle (1234) \rangle$. Then
\[ H \cap N = \langle (13)(24) \rangle \qquad H / (H \cap N) \simeq \ZZ_2 \]
Also see that:
\[NH / N \simeq \ZZ_2 \]
\end{eg}


\begin{Lem}
Take homomorphism $\phi : G_1 \rightarrow G_2$ where $\phi(N_1) = N2$ for $N_i \triangleleft G_i$, ($i = 1,2$). Then there exists a homomorphism $\psi : G_1 / N_1 \rightarrow G_2 / N_2$ defined by:
\[ (N_1x)\psi = N_2(x\phi) \]
for all $x \in G_1$.   
\end{Lem}

\begin{proof}
%If $\psi$ a well defined map? Does $N_2(x\phi)$ depend on the $x$ chosen from $N_1 x$. 
Let $N_1 x = N_1 y$. Then $y = nx$ for some $n \in N_1$. So $y \phi = (nx)\phi = (n\phi)(x\phi) \in N_2 (x\phi)$. Thus $N_2(y\phi) = N_1(x\phi)$. Thus the choice of $x \in N_1$ does not matter and the map is well defined. \\
\\
Now let $g,h \in G_1$. And consider:
\[ (N_1 g)\psi (N_1 h)\psi = N_2(g\phi) N_2(h\phi) = N_2(g\phi)(h\phi) = N_2((gh)\phi) = (N_1(gh))\psi \]
Thus $\psi$ is indeed homomorphic. 
\end{proof}

\begin{thm}[Third Isomorphism Theorem]
Let $M, N \triangleleft G$, and $M \geq N$. Then 
\[ \frac{G/N}{M / N} \simeq G / M \]
(TODO: add diagram 2)
\end{thm}


\begin{proof}
Note $N \triangleleft G$ thus $N \triangleleft M$ since $M \geq N$. So by the correspondence theorem
\[ M / N \trianglelefteq G / N \]
Using the previous lemma, take $G_1 = G$, $N_1 = N$, $G_2 = G / M$, and $N_2 = 1$. Let $\phi : G \rightarrow G / M$ and obtain the corresponding $\psi$ from the lemma. Then $\psi : G / N \rightarrow G / M$ where $Nx \mapsto Mx$ for all $x \in G$. \\
\\
Now $\ker \psi \triangleleft G / N$, and it has the form $K / N$ for some $K \triangleleft G$ (again by the correspondence theorem). Let $Nx \in K / N$, then $(Nx) \psi = M$. But $Mx = M$ iff $x \in M$, so $K / N \leq M / N$.  If $x \in M$ then $(Nx) \psi = M$ and so $Nx \in K / N$. Therefore $K / N \geq M / N$. Hence
\[ \ker \psi  = M/ N \qquad \text{and so} \qquad \frac{G / N}{M / N} \simeq \frac{G}{M} \]

\end{proof}

\begin{Def}
Take $G$ and $\Omega \neq \varnothing$. Assume $g \in G$, and $\alpha \in \Omega$. There is defined a unique $g \cdot \alpha \in \Omega$
\begin{enumerate}
\item $\alpha \cdot 1_G = \alpha$ for all $\alpha \in \Omega$ 
\item $(\alpha \cdot g) \cdot h = \alpha \cdot (gh)$ for all $\alpha \in \Omega$ and $g,h \in G$.   
\end{enumerate}
Then we say $G$ acts on $\Omega$, and $\cdot$ is an action of $G$ on $\Omega$
\end{Def}

\begin{eg}
Take $\Omega \neq \varnothing$, $G \leq \text{Sym}(\Omega)$ then for all $\alpha \in \Omega$ and all $g \in G$, 
\[ \alpha \cdot g  = \alpha ^ g \text{ image of $\alpha$ under $g$} \]
\begin{enumerate}
\item is satisfied by definition of the identity of Sym($\Omega$)
\item is satisfied by the definition of multiplication in Sym($\Omega$)
\end{enumerate}
\end{eg}

\begin{Def}
If $\Omega = G$, then we call is a regular action. 
\end{Def}

\newpage
\section{Tuesday $6^{th}$ March}

\begin{Cor}
$G \simeq S \leq \text{Sym} (\Omega)$. 
\end{Cor}

\begin{Def}
Take a group $G  =\Omega$, then take the group action $\alpha \cdot g = g^{-1} \alpha g$. We call this the conjugation action. 
\end{Def}
\begin{rem}
The group action need not be the same as the the original group multiplication.
\end{rem}
\begin{Def}
For a group $G$ and $X \subseteq G$, consider
\[ Xg = \{ xg : x \in X \} \qquad \forall g \in G \]
Define an action on $\mathcal{P}(G)$ by:
\[ X\cdot g = Xg \]
We call this the action on subsets. \\
\\
If $H \leq G$, take $\Omega = \{ Hx : x \in G\}$. If $x \in \Omega$, then
\[ Xg \in G \qquad \text{since} \quad (Hx)g = H(xg). \]
I.e. right multiplication defines an action of $G$ on $\Omega$. We call this the action on subgroups.
\end{Def}

\begin{Lem}
Take $G$, and a group action of $G$ on some $\Omega$. For $g \in G$ define $\pi_g : \Omega \rightarrow \Omega$ by $\alpha \mapsto \alpha \cdot g$. Then
\[ \pi_g \in \text{Sym} ( \Omega ), \]
and the map $\theta : G \rightarrow \text{Sym} (\Omega)$ given by $g \mapsto \pi_g$ is a homomorphism. Note that $\ker \theta$ is the kernel of the action.
\end{Lem}

(TODO : Insert D3)

\begin{proof}
For all $g,h \in G$:
\begin{align*}
(\alpha) \pi_g \pi_h &= (\alpha \cdot \pi_g) \pi_h \\
                    &= (\alpha \cdot g) \cdot h \\
                    &= \alpha \cdot (gh) \\
                    &= \alpha \pi_{gh}
\end{align*}
Also $\alpha \pi_1 = \alpha \cdot 1 = \alpha$ so $\pi_1$ is the identity. \\
\\
For $g \in G$, $\pi_g \pi_{g^-1} = \pi_1$ so $\pi_g \in \text{Sym}(\Omega)$. 
\[ \theta(g) \theta(h) = \pi_g \pi_h = \pi_{gh} = \theta(gh) \]
Thus $\theta$ is a homomorphism. \\
\\
Lastly, for $g \in G$, $ g \in \ker \theta$ if and only if $\pi_g = \pi_1$. So $\alpha \cdot g = \alpha$ for all $\alpha \in \Omega$: i.e. $g\in $ kernel of the action. 
\end{proof}

\begin{Cor}
$\theta : G \rightarrow \text{Sym} (\Omega)$ is a homomorphism then
\[ K = \ker \theta \triangleleft G \quad \text{and} \quad G / \ker \theta \simeq \text{Im} \theta \]
\end{Cor}

\begin{thm}
If $H \leq G$, and $|G : H| = n < \infty$, there exists $N \triangleleft G$ such that $N \leq H$ and $|G : N| \big| n!$, if $n > 1$ and $|G| \nmid n!$, then $G$ is not simple.
\end{thm}

\begin{proof}
$G$ acts by right multiplication on 
\[ \Omega = \{ Hx : x \in G \} \qquad |\Omega| = n \]
So $\theta : G \rightarrow \text{Sym}(n)$. Take $N = \ker \theta$ and so $G / N \simeq \text{Sym}(\Omega)$. Now
\[ |S_n| = n! \implies |G : N| = |G / N| \big| n! \]
To see that $N \leq H$, let $x \in N$ and $H \in \Omega$. Then 
\[ x \in Hx = H \cdot x = H \]
since $x \in N$. Thus $N \leq H$. \\
\\
If $n > 1$: then $H < G$ and so $N < G$. If $N = 1$ then $|G| = |G / N| \big| n!$. This proves $N > 1$ so $G$ is not simple.
\end{proof}

\begin{Cor}
Take $H \leq G$, with $G$ finite. If $|G:H| = p$ and p is the smallest prime divisor of $|G|$, then $H \triangleleft G$.
\end{Cor}

\begin{rem}
Take $H \leq G$ with $|G:H| < \infty$. Then there exists $N \triangleleft G$ of finite index with $N \leq H$. We call
\[ N = \text{core}_G(H), \]
which is the largest normal subgroup of $G$ contained in $H$. 
\end{rem}

\begin{Lem}
Take $H \leq G$ of finite index. Then
\[ N = \text{core}_G(H) = \bigcap_{x \in G} H^x \]
If $M \triangleleft G$ and $M \leq H$, then $M \leq N$.
\end{Lem}
\begin{proof}
(TODO: exercise)
\end{proof}

\begin{eg}
Take $G = S_4 = \langle (1234), (12) \rangle$. And $H = \langle (13)(24), (34) \rangle \simeq D_4 \leq S_4$. Then $|G:H| = 3|$. Thus there exists a $\theta : G \rightarrow \text{Sym} (3)$. Take
\[ N = \ker \theta \qquad N = \text{Core}_G(H) = H \cap H^{(1234)} \cap H^{(12)} = \langle (12)(34), (14)(23) \rangle \simeq \ZZ_2 \times \ZZ_2 \]
Also $G / N \simeq \text{Sym}(3)$.
\end{eg}


\begin{Def}
We call a group action $(G, \Omega, \cdot)$ transitive if for every two elements $\alpha, \beta \in \Omega$ there exists a $G \in G$ such that $\alpha \cdot g = \beta$. 
\end{Def}
\begin{rem}
i.e. the number of orbits is 1
\end{rem}


\begin{eg}
Take a group $G$ and a regular action. Consider $H \leq G$, and $\Omega = \{ Hx : x \in G \}$. Then the action on right cosets is transitive. 
\end{eg}

\begin{eg}
A regular action on $G$ is not necessarily transitive, because only elements of the same order can be conjugates.
\end{eg}

\begin{Def}
Take a group action $(G, \Omega, \cdot)$. The orbit of this action are of the form $ \{ \alpha \cdot g : g \in G \} \subseteq \Omega$.
Then the action is transitive if and only if the number of orbits is 1.
\end{Def}


\begin{Lem}
$\Omega$ is a disjoint union of orbits. \\
\\
This is analogus to parition of $G$ under cosets of $H \leq G$.
\end{Lem}

\begin{eg}
Conjugation action of $G$ on $\Omega$ is equivalent to conjugacy classes of elements of $G$. I.e. $x \in Z(G) \implies \text{class}(x) = \{x\}$.
\end{eg}

\begin{eg}
$S_3$, $\Omega = \{1,2,3\}$. This action has one orbit, $x = (123)$ and the conjugacy classes are:
\[ \{ e\}, \{ (12),(13),(23) \}, \{ (123),(132) \} \]
\end{eg}

\newpage
\section{Thursday $8^{th}$ March}


\begin{Def}
For a group action $(G , \Omega, \cdot)$ we call
\[ G_\alpha = \{ g \in G : \alpha \cdot g = \alpha \} \]
the stabiliser of $\alpha \in \Omega$ in $G$. 
\end{Def}

\begin{Lem}
$G_\alpha \leq G$. 
\end{Lem}

\begin{eg}
Consider a congucation action $(G, \Omega, \cdot)$. Then 
\[ G_x = \{ g \in G : x^g = x \} = C_G(x) \]
\end{eg}

\begin{eg}
Consider a congugation on a subgroup $X \leq G$, given by $X \cdot g = X^g$. Then
\[ G_x = \{ g \in G : X^g = X \} = N_G(x) \]
\end{eg}

\begin{eg}
Consider an action of $G$ on right cosets of $H \leq G$. i.e. $\Omega = \{ Hx : x \in G\}$. Then the stabiliser of $Hx$ in $G$ is $H^x$. 
\end{eg}


\begin{thm}[Orbit Stabiliser Theorem]
Take a group action $(G, \Omega, \cdot)$. Then let $O_\alpha$ be the orbit of $\alpha \in \Omega$ under $\cdot$. Let $H = G_\alpha$ be the stabiliser of $\alpha$ in $G$. Then there exists a bijection
\[ o_\alpha \leftrightarrow \{ G_\alpha x : x \in G \} \] 
\end{thm}
\begin{rem}
If $G$ is finite then 
\[ |G| = |O_\alpha| | G_\alpha| \]
\end{rem}

\begin{proof}
Define $f : O_\alpha \rightarrow \{ Hx : x \in G \}$ by the following. Take $\beta \in O_\alpha$, and choose $x \in G$ with $\beta = \alpha \cdot x$, and then give
\[ f(\beta) = Hx \]
First consider that if $f(\beta) = Hy$, then we can show that $Hy = Hx$, so $f$ is well defined. Then consider 
\[ Hx = f(\alpha \cdot x) \]
so $f$ is onto. Lastly, take $f(\beta) = f(\gamma)$. Then $\beta = \alpha \cdot x$ and $\gamma = \alpha \cdot y$, so $Hx = Hy$ and thus $y = hx$ for some $h \in H$. Then
\[ \gamma = \alpha \cdot y = \alpha \cdot (hx) = (\alpha \cdot h) \cdot x = \alpha \cdot x = \beta \]
as $h \in H$. Thus $f$ is also injective, and gives us the theorem.
\end{proof}

\begin{Cor}[Fundamental Counting Principle]
Suppose $G$ acts on $\Omega$ and $| O_\alpha | = |G : G_\alpha|$. Then if $G$ is finite, $|O_\alpha| = |G| / |G_\alpha|$ and $|O_\alpha| \big| |G|$. 
\end{Cor}

\begin{Cor}
Take a conjugation action of elements of $G$. Then
\[ |Cl(g) |= |G : C_G(g)| \]
\end{Cor}
\begin{Ex}
Take finite $G$ and suppose every two non-trivial elements of $G$ are conjugates, then $|G| \leq 2$. 
\end{Ex}


(TODO : FIX COR BELOW)
%\begin{Cor}
%Take a conjugation action on the subgroups $H \leq G$. Then 
%\[ X \leq G : # of distinct $G$-conjugates of $X$ = |G : N_G(x)| \]
%\end{Cor}

\begin{Lem}
Let $H,K \leq G$, with $H$ and $K$ finite. Then
\[ |HK| = \frac{|H||K|}{|H \cap K|} \]
\end{Lem}

\begin{proof}
$HK = \{ hk : h \in H, k \in K\}$, $\Omega = \{Hx : x \in H\}$. Then $K$ acts on $\Omega$ by right multiplication.
\end{proof}

\begin{Def}
$G$ is a p-group if $|x| = p^k$ for some $k \geq 0$, for all $x \in G$. Then if $G$ is finite, $|G| = p^n$ for some $n \geq 0$. 
\end{Def}

\begin{Lem}
Take prime $p$ and $G$ a finite $p$-group with $|G| = p^n$. Then
\[ Z(G) > 1 \]
\end{Lem}

\begin{proof}
Let $C_1, C_2, \dots, C_r$ be conjugacy classes of $G = \Omega$. Then
\[ |G| = |C_1| + |C_2| + \cdots + |C_r| \]
Let $C_1 = \{e\}$, so $|C_1| = 1$. Then take $x_i \in C_i$. $|C_i| = |G| / |C_G(x_i)| = p^{j_i}$ where $j_i \geq 0$. If all $j_i \geq 1$ then
\begin{align*}
|G| &\cong 0 \mod p \\
\sum_{i = 1}^r &\cong 1 \mod p
\end{align*}
which is a contradiction. Thus $j_i = 0$ for some $i$, and so $C_G(x_i) = G$ for some $x_i \neq 1$. Thus $Z(G) > 1$.
\end{proof}

\begin{rem}
Lagrange implies $p \big| |Z(G)|$.
\end{rem}

\begin{Def}
For a standard group action, define:
\[ \chi(g) = | \{ \alpha \in \Omega : \alpha \cdot g = \alpha \} | \]
as the number of fixed points under $g$.
\end{Def}

\begin{eg}
Take $G = S_5$ and $\Omega = \{ 1,2,3,4,5\}$. then
\begin{align*}
g &= (123)             &\chi(g) = 2 \\
g &= e                &\chi(g) = 5 \\
g &= (12345)            &\chi(g) = 0
\end{align*}

\end{eg}




\begin{eg}
Take a regular action of $G$ on $\Omega = G$. Then
\[ \chi(g) = \begin{cases} 0 & g \neq 1 \\ |G| & g = 1 \end{cases} \]
\end{eg}

\begin{eg}
Consider a conjugation action of elements. Then
\[ \chi(g) = |C_G(g)| \]
\end{eg}

\begin{thm}[C-F]
Let $G$ act on $\Omega$ with both $G$ and $\Omega$ finite. Then the number of orbits is
\[ \frac{1}{|G|} \sum_{x \in G} \chi(x) \]
\end{thm}

\begin{proof}
Define $S = \{ (\alpha, g) : \alpha \in \Omega, g \in G, \alpha \cdot g = \alpha \}$. Then count the cardinality of this set. \\
\\
First, for each $\alpha \in \Omega$, there exists $|G_\alpha$| elements $g$ such that $(\alpha, g) \in S$. Thus
\[ |S| = \sum_{\alpha \in \Omega} |G_\alpha| \]
Secondly, for each $g \in G$, the number of elements of $\alpha \in \Omega$ paired with $g$ is $\chi(g)$. So
\[ |S| = \sum_{g \in G} \chi(g) \]
Combining these gives:
\[ \frac{1}{|G|} \sum_{g \in G} \chi(g) = \frac{1}{|G|} \sum_{\alpha \in \Omega} |G_\alpha| = \sum_{\alpha \in \Omega} \frac{1}{|O_\alpha|} \]
\end{proof}

\newpage
\section{Monday $12^{th}$ March}
\begin{Cor}
If $G$ is finite and $|\Omega| >1$ then there exists an element $g \in G$ with $\chi(g) = 0$.  
\end{Cor}

\begin{eg}
Conisder $D_4$ acting on $\Omega = \{ 1,2,3,4 \}$. Then $\chi(e) = 4$, $\chi(24) = \chi(13) = 2$ and also other characters vanish. Hence the number of orbits of this action is $\frac{1}{8} (8) = 1$. 
\end{eg}

\begin{eg}
Consider $D_4$ again, this time acting by conjugation. Then $\chi(e) = \chi((13)(24)) = 8$ and all other elements have $\chi(g) = 4$. Thus this action has $\frac{1}{8} 40 = 5$ orbits. 
\end{eg}

\subsection{Sylow Theorems}
\begin{thm}[Lagrange]
If $G$ is finite, and $H \leq G$ then $|H| \, \big| \, |G|$.
\end{thm}
We want to understand a form of converse: if $m \, \big| \, |G|$ then does there exist a subgroup $H \leq G$ with $|H| = m$?

\begin{Ex}
Does there exist a subgroup of $A_4$ of order 6?
\end{Ex}

\begin{Lem}
Let $n = p^a m$. Then $^nC_{p^a} \equiv m \mod p$
\end{Lem}

\begin{proof}
Let $f(x), g(x) \in \ZZ[x]$. $f(x) \equiv g(x) \mod p$ if and only if the coefficients of each $x^j$ are congruent $\mod p$. Observe that $(x + 1)^p \equiv (x^p + 1) \mod p$ since $^pC_j \equiv 0 \mod p$ for $1 \leq j \leq p-1$. Therefore it follows from a simple induction that 
\[ (x+ 1)^{p^a} \equiv (x^{p^a} + 1) \mod p \]
And thus
\[ (x+1)^n = (x+1)^{p^a m} \equiv (x^{p^a} + 1)^m \mod p \]
Then compare coefficients to get
\[ \binom{n}{n - p^a} = \binom{n}{p^a} \equiv \binom{m}{1} \equiv m \mod p \]
\end{proof}

\begin{thm}[Sylow I]
If $G$ is a finite group with $|G| = p^a m$ for some prime $p$ and $m$ with $\gcd(p,m)= 1$. Then $G$ has a subgroup of order $p^a$.
\end{thm}

\begin{proof}
Take $\Omega = \{ X \subset G : |X| = p^a \}$. Then $|\Omega| = ^{|G|}C_{p^a} \equiv m \neq 0 \mod p$. Let $G$ act on $\Omega$ by right multiplication (which is well defined as $|Xg| = |X|$). Then $p$ does not divide $|\Omega|$ so there exists some orbit $O$ where $p$ does not divide $|O|$. Let $x \in O$. By the fundamental counting principle, $|O| |G_x| = |G|$. Considering divisibility of $|G|$ by $p^a$ gives $p^a \big| \, |G_x|$ so $p^a \leq |G_x|$. Now for any $X \in \Omega$ with $X \supseteq \{x\}$ and $y \in X$ and $h \in G_x$, $yh \in Xh = X$. Thus $xG_x \subseteq X$, so $|G_x| = |xG_x| \leq |X| = p^a$, and hence $|G_x| \leq p^a$. Thus $G_x$ has order $p^a$, giving us the desired subgroup.    
\end{proof}

\begin{Def}
Let $G$ have $|G| = p^a m < \infty$ for prime $p$ and $\gcd(p,m) = 1$. Then a \textbf{Sylow-$p$} subgroup of $G$ is any $P \leq G$ such that $|P| = p^a$. We call the set of such subgroups
\[ \text{Syl}_p(G) = \{ P \leq G : |P| = p^a \} \]
\end{Def}


\begin{Cor}
If $|G|$ is finite and $p \big| \, |G|$ for prime $p$ then $G$ has an element of order $p$. 
\end{Cor}

\begin{proof}
There exists some $P \in \text{Syl}_p(G)$ by Sylow I. Choose $x \in P \setminus \{ 1 \}$. Then $|x| = p^a$ for some $a \geq 1$, and so $|x^{p^{a-1}}| = p$. 
\end{proof}

\begin{Def}
A group $P$ is a $p$-group if every element has finite order that is a power of $p$.
\end{Def}

\begin{Cor}
A finite group is a $p$-group if its order is a power of $p$. 
\end{Cor}
\begin{proof}
Apply Cauchy and Lagrange
\end{proof}

\begin{eg}
Take $G = \text{Sym}(4)$. Then $|G| = 24 = 2^3 \cdot 3$. We have 
\begin{align*}
S &= \langle (12), (13)(24) \rangle \\
T &= \langle (13), (12)(34) \rangle 
\end{align*}
As Sylow $2$-groups of $G$, and:
\begin{align*}
A &= \langle (142) \rangle \\
B &= \langle (123) \rangle \\
C &= \langle (243) \rangle \\
D &= \langle (143) \rangle\
\end{align*}
are the Sylow $3$-groups of $G$. 
\end{eg}

\begin{rem}
If $S \leq G$ is a Sylow $p$-subgroup then as $|S^g| = |S|$, $S^g$ is also a sylow $p$-subgroup of $G$. Hence $\text{Syl}_p(G)$ is closed under conjugation. 
\end{rem}

\begin{thm}[Sylow II]
Let $G$ be a finite group. Then $\text{Syl}_p(G)$ is a single conjugacy class of subgroups of $G$. 
\end{thm}













\newpage
\section{Tuesday $13^{th}$ March}

$|G| = p^a m$ with $\gcd(p, m) =1$, then there exists an $H \leq G$ with $|H| = p^a$. $\text{Syl}_p(G)$ is a single conjugacy class of $G$. 

\begin{thm}
$G$ finite, $P \leq G$ a $p$-subgroup and $S \in \text{Syl}_p(G)$ then $P \subseteq S^x$ for some $x \in G$. I.e., every $p$-subgroups is contained in some sylow $p$-subgroup of $G$
\end{thm}

\begin{proof}
Let $\Omega = \{ Sx : x \in G\}$. Let $P$ act on $\Omega$ by right multiplication. 
\[ (Sx) \cdot y = S(xy) \]
$|\Omega| = |G : S|$ is not divisible by $p$ since $S \in \text{Syl}_p(G)$. Thus some orbit of the action of $P$ on $\Omega$ must have size not divisible by $p$. But $P$ is a p-group, so all those orbits have $p$-power size by FCP. Since only $p$-power not divisible by $p$ is $1$, there exists some orbit of size $1$. Therefore $P$ stabilisies $Sx$ for some $x \in G$. Let $y \in P$, so $Sx = Sx \cdot y = S(xy)$. Thus
\[ S^x = x^{-1} S x = x^{-1} S xy = S^x y \]
so $y \in S^x$. Thus $P \subseteq S^x$.
\end{proof}

\begin{proof}[Sylow II]
Take $S \in \text{Syl}_p(G)$. Then $|S^x| = |S|$ only if $S^x \in \text{Syl}_p(G)$. If $P$ is a $p$-subgroup of $G$ then $P \subseteq $ some sylow $p$-group. Thus $P \leq S^x$ for some $x \in G$. But then $|P| = |S^x|$ so $P = S^x$. Therefore there exists just one conjugacy class of Sylow $p$-subgroups. 
\end{proof}

\begin{Cor}
$\# \text{Syl}_p(G) = | G : N_G(P) |$ for $P \in \text{Syl}_p(G)$. Also $\# \text{Syl}_p(G) \big| |G :P|$.
\end{Cor}


\begin{Cor}
Let $S \in \text{Syl}_p(G)$ TFAE. Then
\begin{enumerate}
\item $S \triangleleft G$
\item $S$ is unqiue Sylow $p$-subgroup of $G$
\item Every $p$-subgroup of $G$ is contained in $S$
\item $S$ characteristic subgroup of $G$
\end{enumerate}

\end{Cor}

\begin{proof}
(1) implies (2) by the prior corollary. (2) implies (3) as $S = S^x$. (3) implies (4) as automorphisms fix sylow subgroups. (4) implies (1) trivially. 
\end{proof}

\begin{rem}
This means that we can prove that a group is not simple by showing that $G$ has a unique Sylow $p$-subgroup.
\end{rem}

\begin{eg}
Take $|G| = 360 = 2^3 \cdot 3^2 \cdot 5$. Then $N_p(G) = \# \text{Syl}_p(G)$ thus
\[ N_3(G) \big| 2^3 \cdot 5 \]
Thus 
\[ N_3(G) \in \{ 1,2,4,8,5,10,20,40\} \]
\end{eg}


\begin{thm}[Sylow III]
If $G$ is a finite group, then
\[ N_p(G) = \# \text{Syl}_p(G) \quad \text{and} \quad N_p(G) \cong 1 \mod p \]
\end{thm}

\begin{Lem}
Take finite $G$, $S \in \text{Syl}_p(G)$. Let $P$ be a $p$-subgroup of $N_G(S)$. Then $P \subseteq S$. 
\end{Lem}
\begin{proof}
$S$ a Sylow $p$-subgroup with $S \leq N_P(S)$ implies $S \in \text{Syl}_P(N_G(S))$. But by definition $S \triangleleft N_G(S)$ so $S$ is unique, and thus $P \subseteq S$ by the previous corollary. 
\end{proof}

\begin{cl}
$N_P(S) \leq S$
\end{cl}
\begin{proof}[Claim]
Take $G$; $S \in \text{Syl}_p(G)$. $N_p(S) \leq P$ only if $N_p(S)$ is a $p$-group. Which implies $N_p(S)$ is a $p$-subgroup of $N_G(S)$, and thus $N_P(S) \leq S$. 
\end{proof}

\begin{proof}[Theorem]
Let $P \in \text{Syl}_p(G)$. Let $P$ act by conjugation on $\text{Syl}_p(G)$. Since $P$ stabilises itself under conjugation, $\{P\}$ is an orbit under the action. If there is just one orbit, then we are done. Otherwise, let $S \in \text{Syl}_p(G)$ where $S \neq P$. Let $O$ be the orbit of $S$. Then $|O| = |P : N_P(S)|$. Now $N_P(S) \leq S$ from the claim above, and $N_P(S) \leq P$, so $N_P(S) \leq P \cap S$. But also $x \in P \cap S$ only if $x \in S$, so $x$ normalises $S$, and so $N_P(S) \geq P \cap S$. Therefore $|O| = |P : P\cap S|$. But $S \neq P$, so $|P : P \cap S| = p^e$ for some $e \geq 1$. Thus
\[ N_P(S) = 1 + \sum_i p^{e_i} \quad \text{and} \quad N_P(S) \cong 1 \mod p \]
\end{proof}

\newpage
\section{Thursday $15^{th}$ March}

\begin{eg}
Take $G = \text{Sym}(4) = \langle (1234), (12) \rangle$ (with $|G| = 24 = 2^3 \cdot 3$. Then the factors of $3$ are only 1 and 3, which are both congruent to $1\mod 2 = p$. We can construct three subgroups of order 8:
\begin{align*}
S &= \langle (12), (13)(24) \rangle \\
T &= \langle (24), (14)(23) \rangle \\
U &= \langle (14), (13)(24) \rangle
\end{align*}
thus $n_2 = 3$. Note that $U = S^{(24)}$ and $T = U^{(12)}$ so $T = (S^{(24)})^{(12)}= S^{(241)}$. \\
\\
Similarly, $n_3$ must be a factor of 8, i.e., $n_3 \in \{1,2,4,8\}$. But is must also be congruent to $1 \mod 3$, so it cannot be 2 or 8. Again, we can construct at least two subgroups of order 3, so $n_3 = 4$. 

\end{eg}

\begin{Def}
A group $G$ is simple if and only if its only normal subgroups are $1$ and $G$. 
\end{Def}

\begin{eg}
Cyclic groups of prime order are simple, and are in fact the only simple abelian groups. 
\end{eg}

\begin{eg}
If $G$ is simple, non-abelian, and such that $|G| < 1000$ then there are only five possibilities:
\[ |G| \in \{ 60, 168, 360, 504, 660 \} \]
Most of these are of the form $PSL(2,q) = SL(2,q) / Z(SL(2,q))$.
\end{eg}

\begin{Lem}
Let $G$ be such that $|G| = p^a m$ for prime $p$ with $\gcd(p,m) = 1$. If $G$ is simple then $n_p(G)$ satisfies all of the following:
\begin{enumerate}
\item $n_p | m$
\item $n_p \equiv 1 \mod p$
\item $|G| \, \big| \, n_p!$
\end{enumerate}
\end{Lem}

\begin{proof}
We have:
\begin{enumerate}
\item $|\text{Orbit of Sylow }p\text{-subgroups}| \, \big| \, |G|$. Thus $|O| |N_G(P)| = |G|$ and $|O| \, \big| \, |G:P|$.
\item This is just Sylow III
\item Take $S \in \text{Syl}_p(G)$ and $H = N_G(S)$. Then as $G$ os simple and $1 < S < G$, $S$ is not normal in $G$ so $n_p > 1$. Let $G$ act on $\Omega = \{ Hx : x \in G \}$ by right multiplication. Then $|\Omega| = n_p > 1$. Define a homomorphism $\phi : G \rightarrow \text{Sym}(n_p)$. Then $\ker \phi \triangleleft G$ only if $\ker \phi = \{1\}$. So as $G$ is isomorphic to a subgroup of $\text{Sym}(n_p)$ we must have $|G| \, \big| \, n_p! = |\text{Sym}(n_p)|$. 
\end{enumerate}
\end{proof}

\begin{eg}
Let $|G| = 10^6 = 2^6 \cdot 5^6$. Then $G$ is not simple. To see this, consider $n_5 | 2^6$ and $n_5 \equiv 1 \mod 5$, so $n_5 \in \{1, 16\}$. If $G$ were simple, then $|G| \, \big| \, 16!$, but this is clearly false. Hence $G$ is not simple. 
\end{eg}

\begin{eg}
Let $|G| = 21 = 3 \cdot 7$. $n_7 | 3$ and $n_7 \equiv 1 \mod 7$ thus $n_7 = 1$, so $G$ has a unique sylow $7$-subgroup. $P \in \text{Syl}_y(G)$. Then $G / P \simeq C_3$, thus $G' \leq P$. Suppose $Q \triangleleft G$ and $|Q| = 3$. Then $G' \leq Q$, which implies $G' \leq P \cap Q = \{1\}$. Hence there only exists such a $Q$ if $G$ is abelian. If $G$ is non-abelian, then $G$ has a unique Sylow 7-subgroup and 7 Sylow 3-subgroups. 
\end{eg}

\begin{Lem}
Let $|G| = pq$ for distinct primes $p > q$. Then:
\begin{enumerate}
\item G has a normal Sylow $p$-subgroup; and
\item If $G$ is non-abelian, then $q | p -1$ and $G$ has exactly $p$ sylow $q$-subgroups.
\end{enumerate}
\end{Lem}

\begin{eg}
let $|G| = 12$. By our usual considerations $n_3 \in \{1,4\}$. If $n_3 = 1$, then $G$ has a normal $3$-subgroup. So suppose $n_3 = 4$. Then the 4 Sylow 3-subgroups contain the 8 elements of order $3$ (in $G$), as well as the identity element. There must exist a Sylow 2-subgroup of order $4$, must can only contain the remaining $3$ elements, and so must be unique. 
\end{eg}

\begin{Lem}
If $|G| = p^2 \cdot q$ with distinct primes $p,q$, then either $G$ has a normal Sylow $p$-subgroup or it has a normal $q$-subgroup.
\end{Lem}

\begin{eg}
If $|G| = 30 = 2 \cdot 3 \cdot 5$ then $G$ is not simple.
\end{eg}












\newpage
\section{Monday $19^{th}$ March}

\subsection{Finite Abelian Groups}

Take groups $G,H$. Then define
\[ G \times H = \{ (g ,h) : g \in G \ h \in H \} \]
Call this the external direct product, and it is a group when  $(g,h) \cdot (a,b) = (ga, hb)$. We want to understand when a group $G$ can be written as a direct product $G \simeq H \times K$, for some $H, K \leq G$.  

\begin{thm}
Let the group $G$ have subgroups $G_i$ for $1 \leq i \leq n$ where:
\begin{enumerate}
\item $G_i$ commute with $G_j$ element wise;
\item $G = G_1 G_2 \cdots G_n$; and
\item $G_i \cap (G_1 \cdots G_{i-1}G_{i+1} \cdots G_n) = \{e\}$.
\end{enumerate}
Then $G \simeq G_1 \times G_2 \times \cdots \times G_n$. 
\end{thm}

\begin{proof}
Let $g \in G$. Then by (2) every $g \in G$ can be written $g = g_1 g_2 \cdots g_n$. Suppose that also $g = g_1 ' \cdots g_n '$. Then by (1), $g_i  (g_i')^{-1} = g_1^{-1} g_1' \cdots g_{i-1}^{-1} g_{i-1}' g_{i+1}^{-1} g_{i+1}' \cdots g_n^{-1} g_n'$. The left hand side is in $G_i$ and the other side is in $G_1 \cdots G_{i-1}G_{i+1} \cdots G_n$ so by (3) $g_i (g_i')^{-1} = 1$, or rather $g_i = g_i '$. Hence the representation of $g$ is unique. \\
\\
Now define $\phi ; G \rightarrow G_1 \times G_2 \times \cdots \times G_n$ by $g \mapsto (g_1, g_2, \dots, g_n)$. Exercise: show that $\phi$ is an isomorphism. 
\end{proof}
\begin{rem}
(1) is equivalent to (1'): $G_i \triangleleft G$ for all $1 \leq i \leq n$. 
\end{rem}

\begin{Cor}
If $G = MN$ for $N, M \triangleleft G$ with $M \cap N = \{1\}$. Then $G = M \times N$. 
\end{Cor}

\begin{Def}
Take a finite group $G$, with $S_p \triangleleft G$ for all $ p \big| |G|$. Then we say $G$ \textit{nilpotent}. 
\end{Def}
\begin{rem}
All abelian groups are nilpotent.
\end{rem}


\begin{thm}
Let $G$ be a finite nilpotent group. Then 
\[ G = S_{p_1} \times S_{p_2} \times \cdots S_{p_k} \]
for where $P = \{p_i : 1 \leq i \leq k\}$ list the primes dividing $|G|$.
\end{thm}

\begin{proof}
(1') is satisfied by the definition of nilpotent groups. Then consider $S_{p_1} S_{p_2} \cdots S_{p_k} \leq G$. Note that $| S_p| \big| |S_{p_1} \cdots S_{p_k}|$ for all $p \in P$. Thus
\[ \prod_{p \in P} |S_p| \Big| | S_{p_1} \cdots S_{p_k} | \]
But then $|G| = \prod_{p \in P} |S_p|$ so $G = S_{p_1} \cdots S_{p_k}$. So (2) is satisfied. Let $q \in P$. Then 
\[ \left| \prod_{p \in P \setminus \{ q\}} S_p \right| \quad \text{divides} \quad \prod_{p \in P \setminus \{ q\}} |S_p| \]
But $q$ does not divide $\prod_{p \in P \setminus \{ q\}} |S_p|$ so $q$ does not divide $\left|\prod_{p \in P \setminus \{ q\}} S_p \right|$ either. But $S_q \cap \prod_{p \in P \setminus \{ q\}} S_p \leq S_q$ is a group of $q$-power order. Thus $S_q \cap \prod_{p \in P \setminus \{ q\}} S_p = \{ 1 \}$ and so (3) is satisfied too. Hence applying the previous theorem gives
\[ G = S_{p_1} S_{p_2} \cdots S_{p_k} \]
\end{proof}


\begin{thm}
Let $G$ be a finite abelian $p$ group. Let $C \leq G$ be a cyclic subgroup of maximum possible order. Then $G = C \times B$ for some $B \leq G$.
\end{thm}

\begin{proof}
If $C = G$ then take $B = \{1\}$ and we are done. Otherwise, $C < G$. Take $x \in G \setminus C$ of smallest possible order, with $x \neq 1$. Now $|x^p| < |x|$ so $x^p \in C$. If $C = \langle x^p \rangle $ then $|x| = p|C|$ but this contradicts the choice of $C$. Therefore $x^p = y^p$ for some $y \in C$. But $x \notin C$ so $xy^{-1} \notin C$, so $|x| \leq |xy^{-1}|$. But $(xy^{-1})^p = x^p (y^p)^{-1} = 1$ as $G$ is abelian. Thus $|xy^{-1}| = p$ and so $|x| = p$. \\
\\
Take $X = \langle x \rangle$ and define $\phi : G \rightarrow G / X$. As $|X| = p$, $C \cap X = \{1\}$. Thus $\phi(C) = XC / X \simeq C / (C \cap X) \simeq C$. Hence $\phi |_C$ is an isomorphism. $\phi(C)$ is a cyclic subgroup of maximum order in $G / X$ (as otherwise it would contradict our choice of $C$). Since $|G / X| < |G|$ we can then apply induction on $|G|$ and conclude that $\phi(C)$ is direct factor of $G / X$. To see this: \\
\\
By the correspondence theorem, every subgroup of $G / X$ has the form $B / X$ for some $X \leq B \leq G$. By the inductive hypothesis, we can find $X \leq B \leq G$ such that $G / X \simeq \phi(C) \times (B / X)$. Since $G / X = \phi(C) (B / X)$, $G  = CB$, and also $\phi(C) \cap (B/X) = \{1\}$ and so $C \cap B \subseteq X, C$ so $C \cap B \subseteq C \cap X = \{ 1 \}$. Thus $G = C \times B$.  
\end{proof}



\newpage
\section{Tuesday $27^{th}$ February}

\begin{thm}[Fundamental Theorem of Finite Abelian Groups]
Let $G$ be a finite abelian group. Then $G = C_1 \times C_2 \times \cdots \times  C_n$ where the $C_i$ are cyclic $p_i$-groups for various primes $p_i$. 
\end{thm}

\begin{proof}
Consider an induction argument on the order of $G$. If $G$ is decomposable: i.e. $G = A \times B$ for $A,B < G$, then our by our inductive hypothesis is each of $A$ and $B$ are products of cyclic $p$-groups and so $G$ is too. Otherwise, by our previous theorems $G$ is a (finite abelian) $p$-group. Then our other result implies that $G$ is cyclic.
\end{proof}


\begin{thm}
Let $n_1, n_2, \dots, n_r$ and $m_1, m_2, \dots, m_s$ be non trivial prime powers. Then take $C_{n_1} \times C_{n_2} \times \cdots \times C_{n_r} \simeq C_{m_1} \times C_{m_2} \times \cdots \times C_{m_s}$. Then $r = s$ and, after a suitable renumbering (if necessary), $n_i = m_i$ for all $1 \leq i \leq r$. 
\end{thm}
\begin{proof}
See Krull-Schmidt Theorem.
\end{proof}


\begin{Lem}
Take coprime $n,m \in \NN$. Then $C_{nm} \simeq C_n \times C_m$.
\end{Lem}
\begin{proof}
Exercise
\end{proof}

\begin{Ex}
Every infinite abelian cyclic group is isomorphic to $(\ZZ, +)$. 
\end{Ex}

\begin{Def}
We say a group $G$ is torsion-free if no element of $G$ other than the identity has finite order. 
\end{Def}
\begin{eg}
$(\ZZ, +)$ is torsion free.
\end{eg}

\begin{thm}
A non-trivial torsion-free finitely generated abelian group is isomorphic to the direct sum of a set of infinite cyclic groups
\end{thm}

\begin{proof}
Let $A$ be an appropriate group. Choose a generating set $A = \langle a_1, \dots, a_n \rangle$ with minimal $n$ (which exists as $A$ is finitely generated). $A$ is also torsion free, so we cannot have
\[ m_1 a_1 + m_2 a_2 + \cdots + m_n a_n = 0 \]
for non-vanishing $m_i$, as otherwise we could construct a generating set with fewer than $n$ elements - a contradiction. Now let $A_i = \langle a_i \rangle$. Observe that $A_i \triangleleft A$ as $A$ is abelian. Also see that
\[ A = A_i + A_2 + \cdots + A_n \]
from the definitions of our generating set and $A_i$s, and that 
\[ A_i \cap (A_1 + \cdots + A_{i-1} + A_{i+1} + \cdots A_n) = 0, \]
which can be seen from a similar argument as before. Therefore by our previous result:
\[ A \simeq A_1 \oplus A_2 \oplus \cdots \oplus A_n \]
\end{proof}

\begin{Def}
We call $G$ periodic if every element of $G$ has finite order. 
\end{Def}

\begin{Lem}
Let $G$ be an arbitrary abelian group. The elements of finite order in $G$ form a subgroup $P$ (called the periodic subgroup) and $G / P$ is torsion free. 
\end{Lem}

\begin{proof}
Note that the identity has finite order, so $P \neq \varnothing$. Then take $x,y \in P$. $x^t = y^s = 1$, so $(xy^{-1})^{ts} = 1$ and thus $xy^{-1} \in P$. Hence $P \triangleleft G$. Now consider elements $Pg \in G / P$. If $|Pg| = m < \infty$, then $g^m \in P$, so $g$ has finite order. Thus $g \in P$ and so $Pg = P$ - i.e. the only element of $G / P$ of finite order is the identity and so it is torsion free. 
\end{proof}


\begin{rem}
$P$ as defined above need not be a subgroup unless $G$ is abelian
\end{rem}

\begin{rem}
Elements of infinite order need not generate a subgroup. Consider $b \in G$ with finite order and $c \in G$ of infinite order. Then $bc$ and $c^{-1}$ both have infinite order but $b c c^{-1}$ has finite order. 
\end{rem}

\begin{thm}
Let $G$ be a finitely generated abelian group with periodic subgroup $P$. Then $G = P \oplus T$ where $T$ is torsion free. 
\end{thm}

\begin{proof}
By our previous result, $G / P$ is torsion free. Assume it is non trivial. Then $G / P \simeq \oplus_{n = 1}^n \ZZ_i$ generated by $p_{g_1}, p_{g_2}, \dots, p_{g_n}$. Then let $T =  \langle g_1, g_2, \dots, g_n \rangle$. It is an excerise to show that $T \cap P = 0$, so $T$ is torsion free. Suppose there exists a nontrivial relation
\[ m_1g_1 + \cdots + m_n g_n = b \]
for some $b \in P$. Then $(P_{g_1})^{m_1} \cdots (P_{g_n})^{m_n} = P$ as $Pb = P$. But $G / P$ is torsion free so $m_i = 0$ for all $i$. Thus $b = 0$. So again using our previous result $G \simeq P \times T$. 
\end{proof}

\newpage
\section{Thursday $22^{nd}$ March}

\begin{Cor}
Periodic subgroup of a finitely generated abelian group is finitely generated. 
\end{Cor}
\begin{proof}
$P \simeq G / T$ so we are done. 
\end{proof}


\begin{rem}
We have:
\begin{enumerate} 
\item $P$ is uniquely determined
\item $T$ is determined up to isomorphism $ T \simeq G / P$. 
\item If $G$ is not finitely generared then it is not necesarily the case that $P$ is a direct summand.
\end{enumerate}

\end{rem}

\begin{thm}
A finitely generated abelian group $G$ is isomorphic to the direct sum of finite or infinite cyclic groups.
\end{thm}

\begin{proof}
Take $G = P \times T$. Then expand each of $P$ and $T$. 
\end{proof}

\begin{Def}
The rank of a finitely generated abelian group is the number of infinite direct summands in the direct sum decomposition. 
\end{Def}

\begin{thm}
The rank of a finitely generated abelian group is an invariant
\end{thm}

\begin{Def}
A finite group $G$ has exponent $m$ if $m$ is the smallest integer such that every element of $G$ has order dividing $m$. 
\end{Def}

\begin{Ex}
Take $G$ a finite abelian group. Show that $G$ has an element with order equal to it's exponent.
\end{Ex}

\subsection{Finitely Presented Groups}
We now move to finitely presented groups, and begin with an example.
\begin{eg}
Let 
\[ Q_8 = \left\langle \begin{pmatrix} 0 & 1 \\ -1 & 0 \end{pmatrix} , \begin{pmatrix} 0 & i \\ i & 0 \end{pmatrix} \right\rangle \simeq \langle a, b : a^2 = b^2 = (ab)^2 \rangle \]
\end{eg}

\begin{Def}
Take some set $X \neq \varnothing$ with index set $\Lambda$. So $X = \{ x_\lambda : \lambda \in \Lambda \}$. Then define
\[ X^{-1} = \{ x_\lambda^{-1} : \lambda \in \Lambda \} \] 
\end{Def}

\begin{Def}
A word in $X \cup X^{-1}$ is an ordered set of $n\geq 0$ elements each from $X \cup X^{-1}$ with repetitions allowed. The length of the word is $n$.
\end{Def}

\begin{eg}
Words look like:
\[ x_{\lambda_1}^{\epsilon_1} x_{\lambda_2}^{\epsilon_2} \cdots x_{\lambda_n}^{\epsilon_n} \quad \text{where} \ \epsilon_i = \pm 1 \]
If $n = 0$ the the word is trivial. 
\end{eg}

\begin{eg}
If $X = \{ x , y\}$ then some words are:
\[ xyx^{-1}y, \qquad yx^{-1}yxy^{-1} \]
\end{eg}

\begin{Def}
For words, $u,v,w$ of $X$, we define the product $\cdot$ as:
\[ w \cdot 1 = 1 \cdot w =  w\]
And if $u = x_{\lambda_1}^{\epsilon_1} x_{\lambda_2}^{\epsilon_2} \cdots x_{\lambda_n}^{\epsilon_n}$ and $v = x_{\mu_1}^{\delta_1} x_{\mu_2}^{\delta_2} \cdots x_{\mu_m}^{\delta_m}$ then
\[ u \cdot v =  x_{\lambda_1}^{\epsilon_1} x_{\lambda_2}^{\epsilon_2} \cdots x_{\lambda_n}^{\epsilon_n} x_{\mu_1}^{\delta_1} x_{\mu_2}^{\delta_2} \cdots x_{\mu_m}^{\delta_m} \]
i.e. it is the concatenation of the words
\end{Def}

\begin{rem}
The set of all words on $X \cup X^{-1}$ with this product can be regarded as a semi-group (not a group as there are no inverses, yet), which are objects of study in computer science. 
\end{rem}

\begin{Def}
We say two words $u,v$ on $X \cup X^{-1}$ are adjacent if there exist words $\zeta_1, \zeta_2$ on $X \cup X^{-1}$ and $a \in X \cup X^{-1}$ for which either
\begin{align}
u = \zeta_1 \zeta_2 \qquad &\text{and} \qquad v = \zeta_1 a a^{-1} \zeta_2; \text{or} \\
u = \zeta_1 a a^{-1} \zeta_2 \qquad &\text{and} \qquad v = \zeta_1 \zeta_2 
\end{align}
\end{Def}

\begin{eg}
$x^{-1} x y y^{-1}$ os adjacent to both $yy^{-1}$ and $x^{-1} x$.
\end{eg}

\begin{rem}
$u$ is adjacent to $v$ only if $v$ is adjacent to $u$. 
\end{rem}

\begin{Def}
Let $u,v$ be words on $X \cup X^{-1}$. We call $u$ equivalent to $v$ ($u \sim v$) if there exist some $\zeta_i$ for $1 \leq i \leq n$ on $X \cup X^{-1}$ such that $u = \zeta_1$, $v = \zeta_n$ and $\zeta_i$ is adjacent to $\zeta_{i+1}$ for all $1 \leq i < n$. 
\end{Def}

\begin{Lem}
$\sim$ is an equivalence relation on the set of all words on $X \cup X^{-1}$.
\end{Lem}

\begin{Def}
For words $u, v$ on $X \cup X^{-1}$, with equivalent classes $[u], [v]$ under $\sim$. Then define the product of the classes as
\[ [u] [v] = [uv] \]
\end{Def}

\begin{thm}
The product of classes of words is well defined. The set of classes, with this product forms a group. 
\end{thm}

\begin{proof}
Suppose $[u] = [u']$ and $[v] = [v']$. Then $u \sim u'$ and $v \sim v'$ so 
\[ [uv] = [u'v] = [u'v'] \]
It is an exercise to show that this then satisfies the group axioms.
\end{proof}

\begin{Def}
The free group on a non-empty set $X$ is the set of equivalence classes of words on $X \cup X^{-1}$ with the product given above. Often write this as $F(X)$ or just $F$
\end{Def}

\newpage
\section{Monday $26^{th}$ March}

\begin{Def}
A word in $X \cap X^{-1}$ is called reduced if it has the form $x_{\lambda_1}^{\epsilon_1}x_{\lambda_2}^{\epsilon_2}\cdots x_{\lambda_n}^{\epsilon_n}$ where $x_{\lambda_i}^{\epsilon_i} \neq x_{\lambda_{i+1}}^{-\epsilon_{i+1}}$ for all $i < n-1$. 
\end{Def}

\begin{thm}
Each equivalence class of words in $X \cap X^{-1}$ contains one and only one reduced word. 
\end{thm}

\begin{proof}
Let $w$ be a word on $X \cup X^{-1}$. If $w$ is not reduced, then it is adjacent to a word $u$ which is shorter. Then by induction we obtain the equivalent reduced word for $w$. Hence $[w]$ contains a reduced word. For uniqueness, let $w = a_1 \cdots a_n$. Define $w_0 = 1$, $w_1 = a_1$. Define 
\[ w_{i+1} = \begin{cases} w_{i} a_{i+1} & \text{if the last term of } w_i \neq a_{i+1}^{-1} \\
                          z              & \text{if } w_i = za_{i+1}^{-1} \text{ for some word } z
 \end{cases} \]
Consequently, $w_i$ is reduced and $w_i \sim a_1 \cdots a_i$ for all $0 \leq i \leq n$. Thus $[w_n] = [w]$ and if $w$ is already reduced then $w_n = w$. \\
\\
Let $u$ and $v$ be adjacent words:
\begin{align*}
u &= a_1 \cdots a_i a_{i+1} \cdots a_n \\
v &= a_1 \cdots a_i x x^{-1} a_{i+1} \cdots a_n
\end{align*}
The reduction procedure described previously give $u_j = v_j$ for all $j \leq i$. Now consider two cases: either $u_i$ ends with $x^{-1}$ or it does not. In the first case, $u_i = z x^{-1}$ (note $z$ does not end with $x$), so $v_{i+1} = z$ and $v_{i+2} = zx^{-1} = u_i$. In the second, $v_{i+1} = u_i x$ and $v_{i+2} = u_i$. Thus in both cases $v_{i+2} = u_i$. Then continuing gives $u_{j+i} = v_{j+2 + i}$ for $i < n - i$, and in particular, $u_n = v_{n + 2}$. \\
\\
Now suppose $u$ and $v$ are two reduced words in $[w]$. Then $u \sim v$ by a sequence of adjacent words $k_i$. But then the reduced forms of each of these adjacent words are identical, so $u$ and $v$ must be too. 
\end{proof}

\begin{Cor}
if $|X| = |Y|$ then $F(X) \simeq F(Y)$. I.e., the cardinality of $X$ is an invariant of $F(X)$, and we (sometimes) call this the rank of $F(X)$. 
\end{Cor}

\begin{Lem}
Every free group is torsion free. 
\end{Lem}
\begin{proof}
Suppose $[a] \in F \setminus \{ 1 \}$ has finite order $|[a]| = n$. Take a reduced form of $a = b_r^{-1} \cdots b_1^{-1} a_1 \cdots a_s b_1 \cdots b_r$ where $a_1 \neq a_s^{-1}$ for some $r \geq 0$. Then $a^n =  b_r^{-1} \cdots b_1^{-1} (a_1 \cdots a_s)^n b_1 \cdots b_r \in [a]^n = 1$ is also a reduced word of length $2r + ns = 0$. But this contradicts $n \geq 1$, so $a$ must be the empty word. 
\end{proof}

\begin{rem}
We see:
\begin{enumerate}
\item if $X = \{x\}$ then $F(X)$ is the infinite cyclic group.
\item is $|X| > 1$ then $F(X)$ is non-abelian
\end{enumerate}
\end{rem}

\subsection{Free Generators}

Let $X = \{x,y\}$, and $F = F(X)$. Then $X$ has the property that any reduced word on $X \cap X^{-1}$ which is in the identity class in $F$ must be the empty word. But consider $Y = \{ [x], [x^{-1} y x]\}$. Then $\langle Y \rangle$ contains both $[x]$ and $[y]$ and so equals $F$. Reduced words in $[x], [y^x]$ is the conjugate of a reduced word in $[x]$ and $[y]$.


\begin{Def}
A free basis for a free group $F$ is a generating set for $F$ with the property that the only reduced words in them inside the identity class is the empty word
\end{Def}

Observe that $F$ can have many free bases, but that not every generating set is a free basis.  













\newpage
\section{Tuesday $27^{th}$ March}

\begin{thm}
Let $F = F(X)$ where $X = \{ x_\lambda : \lambda \in \Lambda \}$. Take an arbitrary group $G$. If $\{ g_\lambda : \lambda \in \Lambda \} \subseteq G$ then there exists a unique homomorphism $\phi : F \rightarrow G$ with $x_\lambda \mapsto g_\lambda$ for all $\lambda in \Lambda$. 
\end{thm}

\begin{proof}
Define $\phi_0 : X \rightarrow G$ by $s_\lambda \mapsto g_\lambda$. Take $[w] \in F$ where $w = x^{\epsilon_1}_{\lambda_1} \cdots x^{\epsilon_n}_{\lambda_n}$. Define $\phi : F \rightarrow G$ by $[w] \mapsto g^{\epsilon_1}_{\lambda_1} \cdots g^{\epsilon_n}_{\lambda_n}$. It is instructive to check that $\phi$ is well defined: if $[u] = [v]$ then $\phi([u]) = \phi([v])$, and that it is a homomorphism. Uniqueness then follows from the fact that the images of the reduced words in $X \cup X^{-1}$ agree with the images of $x_\lambda$ specified. 
\end{proof}

\begin{thm}
Take $F_m$ and $F_n$ as free groups of rank $m$ and $n$ respecetively. Then $F_m \simeq F_n$ if and only if $m = n$. 
\end{thm}

\begin{proof}
If $m = n$ then it is easy to show $F_m \simeq F_n$. Consider the converse; suppose $F_m \simeq F_n$. Let $G = \langle g : g^2 = 1 \rangle (\simeq \ZZ_2)$. Consider a homomorphism $\phi : F_m \rightarrow G$. This is completely determined by the images of each $x_i \in F_m$: either $x_i \mapsto g \neq 1$ or $x_i \mapsto g^0 = 1$. Thus the number of nontrivial homomorphisms from $F_m$ onto $G$ is $2^m - 1$ (there is also one trivial one). Suppose $\phi$ is non-trivial. Then $K = \ker \phi \triangleleft F_m$ and $F_m / K \simeq \ZZ_2$ by the first isomorphism theorem. In fact every normal subgroup of index 2 is of the form $\ker \phi$ for some non trivial $\phi$. Therefore $F_m$ has $2^m - 1$ normal subgroups of index 2. Similarly $F_n$ has $2^n - 1$ subgroups of index $2$. Thus as $F_m \simeq F_n$ we have $2^m - 1 = 2^n - 1$ and we are done.
\end{proof}

\begin{Cor}
Every free basis for $F_n$ has precisely $n$ elements. 
\end{Cor}


\begin{thm}
Let $G = \langle g_\lambda : \lambda \in \Lambda \rangle$ have the property that any group $H$ containing a subset $\{h_\lambda : \lambda \in \Lambda \}$ the map $\theta : g_\lambda \mapsto h_\lambda$ can be uniquely extended to a homomorphism of $G$ into $H$. Then $G$ is a free group and $\{g_\lambda ; \lambda \in \Lambda\}$ is a free basis of $G$. 
\end{thm}
\begin{proof}
Choose $H$ to be the free group with free basis $\{ h_\lambda : \lambda \in \Lambda\}$. There exists a unique homomorphism $\phi : H \rightarrow G$ with $h_\lambda \mapsto g_\lambda$. Thus $\theta\phi h_\lambda = h_\lambda$ and $\phi \theta g_\lambda = g_\lambda$. Thus $\theta$ is an isomorphism and so $G$ is the free group with free basis $\{ g_\lambda : \lambda \in \Lambda \}$.
\end{proof}

\begin{Def}
A group $F$ is free on $X \subseteq F$ if, given an arbitrary group $H$ and map $\theta: F \rightarrow H$, there exists a unique homomorphism $\theta' : F \rightarrow G$ extending $\theta$. 
\end{Def}

\begin{thm}
Every group is isomorphic to a factor group of a suitable free group. Every group with $n$ generators is isomorphic to a quotient of $F_n$. 
\end{thm}
\begin{proof}
Let $G = \langle g_\lambda : \lambda \in \Lambda\rangle$ and $X = \{ x_\lambda : \lambda \in \Lambda \}$. Then there is a 1-1 correspondence between $X$ and $G$. Let $F = F(X)$, with free basis $X$. There exists a surjective homomorphism $\phi : F \rightarrow G$ with $x_\lambda \mapsto g_\lambda$. Then $G \simeq F / \ker \phi$. If $|\Lambda| = n$ then $F$ has rank $n$. 
\end{proof}

\begin{Def}
Let $X$ be a set and let $\Delta$ be a set of words on $X \cup X^{-1}$. A group $G$ has \textbf{Generators} $X$ and \textbf{Relators} $\Delta$ if $G \simeq F / R$ where $F = F(X)$ and $R$ is the normal closure of $\Delta$ (i.e., the smallest normal subgroup of $F$ containing $\Delta$). We say $G$ has presentation $\{ X : \Delta \}$ and $G = \langle X : \Delta \rangle$. 
\end{Def}

\begin{eg}
Take $G = \ZZ_6 = \langle x : x^6 = 1\rangle$. Then $G \simeq F_1 / \llangle x^6 \rrangle^G$. $x^6$ is a relator. $x^6 = 1$ is a relation.
\end{eg}

\begin{eg}
$G = D_n = \langle x,y : x^n = 1, y^2 = 1, x^y = x^{-1} \rangle$. Then $G \simeq F_2 / \llangle x^n, y^2, x^y x \rrangle^G$.
\end{eg}

\begin{eg}
$Q_8 = \langle x,y : x^2 = y^2, (xy)^2 = y^2 \rangle$. Then $Q_8 \simeq F_2 / \llangle x^2y^{-2}, (xy)^2y^{-2} \rrangle^G$. 
\end{eg}





\newpage
\section{Thursday $29^{th}$ March}

\subsection{Group Presentations}
\begin{eg}
Consider 
\[ Q_8 = \left\langle a = \begin{pmatrix} 0 & 1 \\ -1 & 0 \end{pmatrix}, b = \begin{pmatrix} 0 & i \\ i & 0 \end{pmatrix} \right\rangle \leq GL(2,\CC) \]
Then $a^2 = b^2 = (ab^2)$. Now let $F$ be the free group of rank 2: $F = Free\{ x, y\}$. Define $\phi: F \mapsto Q_8$ by $x \mapsto a$ and $y \mapsto b$. Then $\phi( y^2x^{-2}) = 1$ and $\phi( (xy)^2y^{-2}) = 1$ so $x^2y^{-2}, (xy)^2y^{-2} \in \ker \phi$. Set $K = \langle x^2y^{-2}, (xy)^2y^{-2} \rangle^F$. We claim $K = \ker phi$. Thus we can write 
\[ Q_8 = \langle x,y : x^2 = y^2, (xy)^2 = y^2 \rangle \]
\end{eg}

\begin{Lem}
Every group has presentation (although not necessarily finite) 
\end{Lem}

\begin{proof}
Let $G = \langle X \rangle$. Then $G$ is isomorphic to some quotient of $F = F(X)$. Declare any homomorphism $\theta : F \rightarrow G$. Let $R \subseteq F$ such that $\ker \phi = \langle R \rangle^F$. Now $G \simeq F / \ker \theta$ and so $G$ has presentation $\{ X : R \}$. 
\end{proof}

\begin{thm}[Von Dyck's Theorem (1882)]
Let $G = \langle X : W_1 \rangle$ and let $H = \langle X : W_2 \rangle$ where $W_1 \subseteq W_2$. Then $H$ is isomorphic to a factor group of $G$. 
\end{thm}

\begin{proof}
Let $K_1 = \langle W_1 \rangle^F$ and $K_2 = \langle W_2 \rangle^F$. Note that $K_1 \leq K2$, and that $G \simeq F / K_1$ and $H \simeq F / K_2$. Using the third isomorphism theorem, we then have:
\[ H \simeq \frac{F}{K_2} \simeq \frac{F / K_1}{K_2 / K_1} \simeq \frac{G}{N} \]
\end{proof}

\begin{eg}
Let $G = \langle a,b : a^2 = b^2 \rangle$ and $H = \langle a,b : a^2 = b^2, (ab)^2 = b^2 \rangle$. Then $H \simeq Q_8$ is a quotient of $G$. So is $H_2 = \langle a,b : a^2 = b^2, b^2 = 1 \rangle$ (this is called the infinite dihedral group). 
\end{eg}

Our strategy for using Von Dyck's theorem is: 
\begin{enumerate}
\item Show $|G| \leq n$
\item Exhibit quotient $H$ of order at least $n$
\item Deduce $|G| = n$
\end{enumerate}
For $G = \langle X : R \rangle$ and $H = \langle X : R, \dots \rangle$. 


\begin{eg}
Define $G = \langle a,b,c,d : ab = c, bc = d, cd = a, da = b \rangle$. We show that $|G| \leq 5$ by consideration of the relations. Then define $\phi : F = \langle a,b,c,d\rangle \rightarrow \ZZ / 5 \simeq \ZZ_5 \simeq H$ by 
\begin{align*}
a &\mapsto [1] \\
b &\mapsto [3] \\
c &\mapsto [4] \\
d &\mapsto [2]
\end{align*}
It is clear that $abc^{-1}, \dots dab^{-1} \in \ker\phi$ so by Von Dyck's $H$ is isomorphic to a quotient of $G$. $|G| \geq 5$ and so we are done. 
\end{eg}


\begin{eg}
Let $G = \langle a,b : a^4 = b^2, b^{-1} a b = a^{-1} \rangle$. We note that $a^4$ is central in $G$; it is a power of both $a$ and $b$ and so commute with both. Then $a^4 = b^{-1} a^4 b = (b^{-1} a b)^{-1} = (a^{-1})^4$, so $a^8 = 1$. Hence $\langle a \rangle^G \triangleleft G$. Note that $|\langle a \rangle| \leq 8$ so $|G| \leq 16$. \\
\\
Now, let 
\[ H = \left\langle A = \begin{pmatrix} 3 & 3 \\ -3 & 3 \end{pmatrix}, B = \begin{pmatrix} 0 & 4 \\ 4 & 0 \end{pmatrix} \right\rangle \leq GL(2,17) \]
We claim $|H| = 16$. We also claim that $A^4 = B^2$ and $B^{-1} A B = A^{-1}$. Thus we can use Von Dyck, and $|G| = 16$. 
\end{eg}


\newpage
\section{Monday $16^{th}$ April}

\begin{Lem}
$D_n$ is the dihedral group of order $2n$ (for $n \geq 3$) then
\[ D_n \simeq \langle s,t : s^n = 1, t^2 = 1, s^t = s^{-1} \rangle \]

\end{Lem}

\begin{thm}
Let $G$ be a finite group generated by 2 involutions (elements of order $2$). Then $G \simeq D_n$ for some $n$. 
\end{thm}

\begin{proof}
If $G$ is finite, then $|ab| = n$ for involutions $a,b$. Let $s = ab$. Then
\[ asa = aaba = ba = (ab)^{-1} = s^{-1} \]
Hence:
\[ |\langle a,b \rangle| = |\langle a,s \rangle| = m \] for some $m$. Suppose for a contradiction that $as^i = 1$ for some $i \geq 0$. Choose minimal $i$ with $as^i = 1$. $i \neq 0$ and $a \neq 1$. But also $i \neq 1$ as else $1 = as = aab = b$. But also $1 = as^i = aabs^{i-1} = bs^{i-1}$. Then conjugation by $b$ gives
\[ 1 = s^{i-1}b = s^{i-2} a bb = s^{i-2} a \]
Then conjugation by $a$ gives $1  = as^{i-2}$ which contradicts the minimality of $i$. Thus $as^i neq 1$ for all $i \geq 0$, and so $as^i \neq s^j$ for all $i \neq j$. Thus
\[ \langle a,b \rangle  \geq \langle s \rangle \cup a \langle s \rangle \]
and so $|\langle a,b \rangle | = |\langle a,s\rangle | \geq 2n$. Now define 
\[ H = \{ a^j s^i : 0 \leq j < 2, 0 \leq i < n \} \leq G \]
Then $|H| = 2n$ and $\langle a, s \rangle \leq H$ so $|\langle a,b \rangle| = |langle a,s\rangle| = 2n$, and so $\langle a,b \rangle \simeq D_n$. 
\end{proof}


\begin{thm}
Let $p$ be prime. Then every group $G$ of order $2p$ is either cyclic or dihedral. 
\end{thm}

\begin{proof}
If $p = 2$, then $|G| = 4$ for which the claim is true. Otherwise, $p$ is odd and so $G$ has an element $s$ of order $p$ and an element $t$ of order $2$. Let $H = \langle s \rangle$, so $|G:H| = 2$ and $H \triangleleft G$. Thus $t^{-1} s t = tst = s^i$ for some $i$. Thus $s = t^2 s t^2 = t s^i t = s^{i^2}$, and so $i^2 \equiv 1 \mod p$. But as $p$ is prime, $1 \equiv \pm 1 \mod p$. Therefor $tst = s$ or $tst = s^{-1}$. In the first case, 
\[ \langle s,t : tsts6{-1}, t^2, s^p \rangle \simeq \ZZ_{2p} \]
In the second, 
\[ \langle s,t : tsts, t^2, s^p \rangle \simeq D_p \]
\end{proof}

\begin{thm}
If $|G| = pq$ for distinct primes $p > q$. Either
\begin{enumerate}
\item $G$ is cyclic 
\item $G = \langle a,b : a^p = 1, b^q = 1, b^a = b^m \rangle$ where $m^q \equiv 1 \mod p$ and $m \neq 1 \mod p$
\end{enumerate}
And, if $q$ does not divide $p - 1$, then the second case does not exist. 
\end{thm}


\subsection{Classification of 2-generated groups of order 8}
Let $x \in G$ of maximal order $m$. Then $m \in \{ 2,4,8 \}$ by Lagrange. If $m = 8$, then $G$ is cyclic so $m \neq 8$. If $m = 2$, then $G$ is abelian. Suppose $m = 4$ let $H = \langle x \rangle$ so $|H| = 4$. Then $H \triangleleft G$. Let $y \in G / H$. Then $y^2 \in H$, and $x^y \in H$ and $H$ is normal in $G$. Observe this means $y^2 = x^i$ and $x^y = x^j$ for some $i,j \in \{0,1,2,3\}$. if $i = 1$ or $i = 3$, then $|y| = 8$ which is a contradiction, so $i \in \{0,2\}$. Also $j \in \{1,3\}$ as the orders are the same for conjugate pairs. Thus there are 4 posibilities:
\begin{enumerate}
\item $R = \{ x^4, y^2, x^y = x\}$ 
\item $R = \{ x^4, y^2 = x^2, x^y = x\}$ 
\item $R = \{ x^4, y^2, x^y = x^3\}$ 
\item $R = \{ x^4, y^2 = x^2, x^y = x^3\}$ 
\end{enumerate}
The first two cases give groups isomorphic to $\ZZ_2 \times \ZZ_4$. The third gives $D_4$, and the final case gives the Quaternion group. The last two cases can be seen by applications of von Dyck's theorem. 
 
 
\subsection{Free Abelian Groups}

Take an abelian group $G$. Take $x_1, \dots x_r \in G$. Then any integral linear combination (i.e, a linear combination over $\ZZ$) is an element of $G$. Define a basis for $G$ as a generating set $X$ such that any finite sequence of elements of $X$, the only vanishing integral linear combination is the trivial one. If an abelian group has a basis, then it is a free abelian group.

\begin{eg}
Consider $\ZZ \oplus \cdots \oplus \ZZ$ (with $n$ terms). This has natural basis $e_i = (0, \dots, 1, \dots, 0)$ with the one in the $i^{th}$ position.
\end{eg}







\newpage

\section{Tuesday $17^{th}$ April}

\subsubsection{Subgroups of $\ZZ^n$}

Take $H \leq \ZZ^n$. Then $H$ is generated by at most $n$ elements, call this number $m$. We can represent $H$ as an $m\times n$ matrix $A$ whose rows generate $H$. 

\begin{Def}
$S(A)$ is the set of all linear combinations of the rows of $A$. i.e.,
\[ S(A) = \{ uA : u \in \ZZ^n{}^* \} \]
\end{Def}

We want to solve the decidability of membership in $H$, or equivalently membership in $S(A)$ for a given $m \times n$ matrix $A$. Our approach is to use row-equivalence to get a matrix $B$ with $S(A) = S(B)$, but which is easier to describe.

\begin{Def}
Two matrices $A$ and $B$ are \textbf{Row-equivalent} (and write $A \sim B$) if one can be transformed to the other using the operations:
\begin{enumerate}
\item Interchang  rows;
\item Multiplying a row by -1; or
\item Adding an integral multiple of one row to another (distinct) row.
\end{enumerate}
We call these the \textbf{Integral Row Operators}.
\end{Def}
 
\begin{Lem}
If $A$ is row-equivalent to $B$ then $S(A) = S(B)$, and the subgroups generated by $A$ and $B$ are identical.
\end{Lem}

\begin{Def}
$A$ is in \textbf{Row Hermite Normal Form} if:
\begin{enumerate}
\item First $r$ rows of $A$ are non-zero;
\item For $1 \leq i \leq r$, the index of the first non-zero entry in row i, $j_i$, satisfies $j_1 < j_2 < j_3 < \dots j_r$;
\item $A_{i, j_i} > 0$ for $1 \leq i \leq r$; and
\item If $1 \leq k < l < r$ then $0 \leq A_{k,j_k} < A_{l,j_l}$.
\end{enumerate}
\end{Def}

\begin{eg}
The matrix:
\[ A = \begin{bmatrix} 2 & 1 & 5 & 0 & -1 & 1 & -1 \\
                       0 & 3 & -1 & 2 & 4 & 0 & 2 \\
                       0 & 0 & 0 & 4 & 7 & 1 & 8 \\
                       0 & 0 & 0 & 0 & 0 & 2 & 5 
 \end{bmatrix} \]
is in row-hermite form. Consider $v = (6,0,16,6,4,13,31)$, is $v \in S(A)$? This is equivalent to there being a $u = (a,b,c,d) \in \ZZ^4$ such that $uA = v$. Thus $2a = b$, $a + 3b = 0$, $2b + 4c = b$, $a + c + 2d = 13$. Solving these shows that indeed $v \in S(A)$.
\end{eg}

\begin{thm}
Given an integral matrix $B$, there exists a unique matrix $A$ in row hermite normal form with $B \sim A$. 
\end{thm}

\begin{eg}
Take $G = \ZZ^4 = \langle g_1, g_2, g_3, g_4 \rangle$, and $H = \langle g_1 + 2g_2 + g_4, 3g_1 + g_2 + 2g_3 - g_4, -2g_1 + g_2 + 4g_4 \rangle$. Then let $v = 7g_1 + 6g_3 - 3g_4$. Is $v \in H$? We take the representation of $H$:
\[ B = \begin{bmatrix} 1 & 0 & 2 & 1 \\ 3 & 1 & 2 & -1 \\ -2 & 1 & 0 & 4 \end{bmatrix} \sim \begin{bmatrix} 1 & 0 & 2 & 1 \\ 0 & 1 & 4 & 6 \\ 0 & 0 & 8 & 10 \end{bmatrix} = A\]
Then $(7, 0, -1) A = v$ so $v \in S(A) = S(B)$. 
\end{eg}

\subsection{Abelian Quotients}

Recall for group $G$, $G' = \langle [x,y] : x,y \in G \rangle$ is called the derived group of $G$
\begin{Lem}
$G' \triangleleft G$. $H = G / G'$ is abelian. If $N \triangleleft G$, $G / N$ is abelian if and only if $N \geq G'$. Thus $G / G'$ is the largest abelian quotient of $G$.  
\end{Lem}

\begin{Lem}
For a given $G = \langle \{x_1,\dots, x_r\} : R \rangle$, define $C = \{ [ x_i, x_j ] : i \leq i < j \leq r \}$. Then $G_{ab} = \langle X : R, C \rangle$. 
\end{Lem}

\begin{proof}
It sufficies to prove that $G'$ coincides with the normal closure $\overline{C}$ of $C$ in $G$. Since the generators of $G_{ab}$ all commute, $G_{ab}$ is abelian and so $G' \subseteq \overline{C}$. But by definition $\overline{C} \subseteq G'$ so $\overline{C} = G'$. Thus $G_{ab} = \langle X : R,C \rangle$.  
\end{proof}

\subsection{Finitely Generated Abelian Groups}
For an abelian group $G = \langle g_1, \dots, g_n\rangle$, define a surjection $f : \ZZ^n \rightarrow G$ by:
\[ (a_1, \dots a_n) \mapsto a_1 g_1 + \cdots + a_n g_n \]
Observes that $f$ is a homomorphism, and that
\[ G \simeq \ZZ^n / \ker f \]
Let $H = \ker f$. $H \leq \ZZ^n$ so $H$ is finitely generated too, and so we can represent $H$ by a matrix $A$ where $H = S(A)$. 

\begin{eg}
$f: \ZZ^5 \rightarrow \ZZ_2 \oplus \ZZ_4 \oplus \ZZ_{12} \oplus \ZZ \oplus \ZZ$ with 
\[ (a,b,c,d,e) \mapsto ([a]_2, [b]_4, [c]_{12}, d, e) \]
Then $ker f = \{ (a,b,c,d,e): 2 | a, 4 | b, 12 | c, d = e = 0 \}$. Set 
\[ A = \begin{bmatrix}
 2 & 0 & 0 & 0 & 0 \\
 0 & 4 & 0 & 0 & 0 \\
 0 & 0 & 12 & 0 & 0
 \end{bmatrix} \]
Then $H = S(A)$, and
\[ \ZZ^5 / H \simeq \ZZ_2 \oplus \ZZ_4 \oplus \ZZ_{12} \oplus \ZZ \oplus \ZZ \]
\end{eg}





\section{Thursday $19^{th}$ April}

\begin{Def}
Two matrices $A$ and $B$ are equivalent (over $\ZZ$) if one can be obtained from the other using elementary row and column operations.  We write $A \sim B$. 
\end{Def}

\begin{rem}
This is equivalent to there being matrices $P$ and $Q$ with $A = PBQ$ with $\det P, \det Q = \pm 1$. 
\end{rem}

\begin{Lem}
$A \sim B$ if and only if 
\[ \frac{\ZZ^n}{S(A)} \simeq \frac{\ZZ^n}{S(B)} \]
\end{Lem}

\begin{Def}
A $m\times n$ integral matrix $A$ is in \textbf{Smith Normal Form} if for some $k \geq 0$ the entries $d_i = A_{ii}$ are positive for $1 \leq i \leq k$ and $d_i | d_{i+1}$. 
\end{Def}

\begin{Lem}
If 
\[A = \begin{bmatrix} d_1 \\
                        & d_2 \\
                        & &\ddots \\
                        & &       & d_k \\
                        & & &          & \ddots \\
                        & & & &             & 0 \\
                        & & & & &                & \ddots \\
                        & & & & & &                   & 0
 \end{bmatrix} \]
Then 
\[ \frac{\ZZ^n}{S(A)} \simeq \ZZ_{d_1} \oplus \ZZ^{d_2} \oplus \cdots \oplus \ZZ_{d_k} \oplus \ZZ^{n - k} \]
\end{Lem}

\begin{Lem}[Smith Normal Form Construction]
Let $M$ be an $l \times m$ matrix over $\ZZ$ (with $l \geq n$). Then there exists invertible matrices $P \in GL(n,\ZZ)$ and $Q \in GL(l,\ZZ)$ such that $D = QMP$ is diagonal with non-negative integer entries $d_1, \dots, d_n$ with $d_i | d_{i+1}$ for all $i \leq n-1$. The matrix $D$ is in Smith Normal Form.
\end{Lem}

\begin{eg}
\[ A = \begin{pmatrix} 4 & 2 & 1 & 8 \\
                    -4 & 4 & 2 & -8 \\
                    4 & -1 & 1 & 2 \\
                    4 & 5 & 4 & 2
 \end{pmatrix} \sim \begin{pmatrix} 1 \\
                                    & 3 \\
                                    & & 12 \\
                                    & & & 0 
 \end{pmatrix}
 \]
\end{eg}

\begin{eg}
Take $G = \langle x,y : (xy^3x^{-2})^2, y^{-1} x^2 y^2 \rangle$. What is $G / G'$?
\begin{align*}
\frac{G}{G'} &= \langle x,y : \dots, [x,y] = 1 \rangle \\
             &= \langle x,y : 2x + 6y - 4x, -y + 2x + 2y \rangle  \\
             &= \langle x,y : -2x + 6y, 2x + y \rangle \\
             &\simeq \ZZ^ / \langle -2x + 6y, 2x + y \rangle 
\end{align*}
The matrix $A = \begin{pmatrix} -2 & 6 \\ 2 & 1 \end{pmatrix}$ is the relation matrix describing $H$, i.e. $S(A) = H$. $A$ has Smith Normal Form $B = \begin{pmatrix} 1 & 0 \\ 0 & 14 \end{pmatrix}$. Thus
\[ \frac{G}{G'} \simeq \frac{\ZZ^2}{S(A)} \simeq \frac{\ZZ^2}{S(B)} \simeq \ZZ_{14} \] 
\end{eg}

\begin{eg}
Take $G = \langle a,b,c : a^2c^{-1}, bc^2b, cab^4 \rangle$. Then 
\[ \frac{G}{G'} \simeq \frac{\ZZ^3}{S(A)} \qquad \text{where} \ A = \begin{pmatrix} 2 & 0 & -1 \\
  0 & 2 & 2 \\ 1 & 4 & 1
 \end{pmatrix} \sim B = \begin{pmatrix} 1 \\ & 1 \\ & & 10 \end{pmatrix} \]
 Thus $G / G' \simeq \ZZ_{10}$
\end{eg}


\begin{eg}
Take $G = \langle x,y,z : (xyz^{-1})^2, (x^{-1}y^2z)^2, (xy^{-2}z^{-1})^2 \rangle$ Then 
\[ A = \begin{pmatrix} 2 & 2 & -2 \\ -2 & 4 & 2 \\ 2 & -4 & -2 \end{pmatrix} \sim \begin{pmatrix} 2 \\ & 6 \\ & & 0 \end{pmatrix} \]
So $G / G' \simeq \ZZ_2 \times \ZZ_6 \simeq \ZZ$. 
\end{eg}

\begin{thm}[Basis Theorem for Finitely Generated Abelian Groups]
Given a finitely generated abelian group $G$, there exist integers $k,n \geq 0$ and $d_i \geq 2$ for $1 \leq i \leq k$ where $d_i | d_{i+1}$ for $1 \leq i \leq k$, such that 
\[ G \simeq \ZZ_{d_1} \oplus \cdots \oplus \ZZ_{d_k} \oplus \ZZ^{n - k} \]
Note $k,n,d_i$ are determined with the isomorphism type of $G$.
\end{thm}

\begin{eg}
If $G = \ZZ_4 \oplus \ZZ_4 \oplus \ZZ_4 \oplus \ZZ_3 \oplus \ZZ_9$ then $G \simeq \ZZ_4 \oplus \ZZ_{12} \oplus \ZZ_{36}$ 
\end{eg}

\subsection{Deficiency of Presentations}

\begin{Def}
Let $G = \langle X : R \rangle$ with $|X|, |R| < \infty$. Define the \textbf{Deficiency} of $G$ as $\text{Def}(G) = |X| - |R|$
\end{Def}

\begin{eg}
We have:
\begin{enumerate}
\item $G = \langle x : x^2 \rangle$ has deficiency of 0
\item $H = \langle x : x^2, x^4 \rangle \simeq G$ has deficiency $-1$ 
\item $A = \langle x : x^6 \rangle$ has deficiency of 0
\item $B = \langle x, y : x^2, y^3, [x,y]\rangle \simeq A$ has deficiency of $-1$
\item $P = \langle x,y : x = (xy)^3, y = (xy)^4 \rangle$ has deficiency of 0
\end{enumerate}
\end{eg}

\begin{Lem}
Take $G = \langle X : R \rangle$ with $X$ and $R$ finite. If $G$ is finite then $|X| \leq |R|$. 
\end{Lem}

\begin{proof}
Assume $|X| > |R|$. Then the relation matrix for this presentation has fewer rows than columns The number of $d_i$ for $G_{ab}$ is less than $|X|$. Thus there exists one infinite cyclic factor in $G / G'$. Thus $G_{ab}$ is infinite, and so $G$ must be too. 
\end{proof}

\begin{Cor}
Every group with $|X| > |R|$ (i.e., positive deficiency) is infinite.
\end{Cor}

\begin{rem}
The converse is not true, i.e., if a group has negative deficiency it need not be finite.
\end{rem}




\newpage
\section{Monday $24^{rd}$ April}


\subsection{Residually Finite Groups}

\begin{Def}
Let $P$ be a property, and say $G$ has property $P$ \textbf{residually} if for all $x \in G \setminus \{1\}$, there exists a normal subgroup $N_x \triangleleft G$ such that $x \notin N_x$, and $g / N_x$ has property $P$. 
\end{Def}
\begin{rem}
We will only consider the property that $P$ is finite. I.e., $G$ is residually finite if for all $x \in G \setminus \{1\}$ there exists a normal subgroup $N_x$ such that $x \notin N_x$ and $|G : N_x| < \infty$. 
\end{rem}

\begin{thm}
Every free group is residually finite.
\end{thm}

\begin{proof}
Let $F$ have free basis $\{ x_\lambda : \lambda \in \Lambda \}$. Take a reduced word $x = x_{\lambda_1}^{\epsilon_1} \cdots x_{\lambda_n}^{\epsilon_n}$. Let $S = \text{Sym}(n+1)$. Define a homomorphism $\phi : F \rightarrow S$. Let $x_{\lambda_i} \mapsto \sigma_{\lambda_i} \in S$ where $\sigma_{\lambda_i}^{\epsilon_i}$ maps $i$ to $i+1$. Then $\phi(x_\lambda) = 1$ if $\lambda \notin \{\lambda_1, \dots, \lambda_n \}$. Thus $\sigma_{\lambda_i}$ maps $i = i+1$ if $\epsilon_i =1$ and $i+1$ to $i$ if $\epsilon_i = -1$. This map is injective, as otherwise we get a contradiction on the reduced words. By the universal property, this map then extended to a (unique) homomorphism $\phi : F \rightarrow S$. It is then easy to check that $x \notin \ker \phi$ and $F / \ker \phi \simeq S$ and so is finite.
\end{proof}

\begin{Cor}
The intersection of all subgroups of finite index in a free group is $\{1\}$.
\end{Cor}

\begin{proof}
Let $x$ be in the intersection of all subgroups of finite index. If $x \neq 1$, then there exists a normal subgroup $N_x$ of finite index not containing $x$, a contradiction.
\end{proof}


\subsection{Structure of (Finite) Groups}

Recall that if $H \leq G$ and $N \triangleleft G$ then $HN \leq G$, and $H / (H \cap N) \simeq NH / N$ by the second Isomorphism Theorem. If $N,M \triangleleft G$ and $N \leq M$ then $\frac{G / N}{M / N} \simeq G / M$ by the correspondence theorem.  

\begin{Def}
A \textbf{Subnormal Series} of a group $G$, is a finite series of subgroups $G = G_0 \triangleright G_1 \triangleright G_2 \triangleright \cdots \triangleright G_r = 1$.
\end{Def}

\begin{eg}
Take $G = S_3$. Then a subnormal series is $S_3 \triangleright \langle (123) \rangle \triangleright \{e\}$. If $G = S_4$, we could take
\[ G \triangleright V = \langle (12)(34), (13)(24) \rangle \triangleright U = \langle (12)(34) \rangle \triangleright 1 \] 
\end{eg}

\begin{Def}
If $G$ has two subnormal series, then we say that the second is a refinement of the first if each member of the first is also a member of the second.
\end{Def}

\begin{eg}
The subnormal series $S_4 \triangleright A_4 \triangleright V \triangleright U \triangleright 1$ is a refinement of previous example.
\end{eg}

\begin{Def}

\[ \{ G_{i-1} / G_i : 1 \leq i \leq r \} \quad \text{and} \quad \{ H_{i-1} / H_i : 1 \leq i \leq s \} \]
such that corresponding factor groups are isomorphic.
\end{Def}

\begin{eg}
The group $\ZZ_6$ has isomorphic subnormal series:
\begin{align*}
\ZZ_6 &\triangleright \ZZ_3 \triangleright 1 \\
\ZZ_6 &\triangleright \ZZ_2 \triangleright 1
\end{align*}
\end{eg}

\begin{thm}[Refinement Theorem]
Any two subnormal series of a given group $G$ have isomorphic refinements.
\end{thm}

%\begin{rem}
%This is sometimes called the butterfly lemma or Zassenhaus Lemma.
%\end{rem}


\begin{Def}
A composition series of $G$ is a subnormal series without repetitions which cannot be further refined. 
\end{Def}

\begin{prop}
If $G \triangleright G_1 \triangleright \cdots G_r = 1$ is a composition series then $G_i / G_{i+1}$ is simple.
\end{prop}

\begin{eg}
The most refined subnormal series of $S_4$ from the previous example is a composition series.
\end{eg}




\newpage
\section{Tuesday $24^{th}$ April}

\begin{Lem}[Zassenhaus / Butterfly Lemma]
Given $A, B \leq G$ and $X \triangleleft A$ and $Y \triangleleft B$: 
\begin{enumerate}
\item $X(A \cap Y) \triangleleft X (A \cap B)$;
\item $Y(B \cap X) \triangleleft Y ( A \cap B)$; and
\item $\dfrac{X (A \cap B)}{X(A \cap Y)} \simeq \dfrac{Y(A\cap B)}{Y(B \cap X)}$
\end{enumerate}
\end{Lem}

\begin{proof}
We leave the proof of (1) and (2) as an exercise. Since $A \cap B \leq A$ and $X \triangleleft A$, $X(A \cap B) \leq A$. But also $X \triangleleft X (X \cap B)$, so by the second isomorphism theorem, 
\[ \frac{X (A \cap B)}{X} \simeq \frac{A \cap B}{X \cap B} \]
Then applying the third isomorphism theorem gives us that $\frac{A \cap B}{(X \cap B)(A \cap Y)}$ is isomorphic to a factor group of $\frac{A \cap B}{X \cap B} \simeq \frac{X (A \cap B)}{X}$. If we can define $\phi : \frac{X(A \cap B)}{X} \rightarrow \frac{A \cap B}{(A \cap Y)(B \cap X)}$ such that $\ker \phi = \frac{X (A \cap Y)}{X}$, then:
\[ \frac{X(A\cap B}{X(A \cap Y)} \simeq \frac{A \cap B}{(X \cap B)(A \cap Y)} \simeq \frac{Y(A \cap B)}{Y(B \cap X)} \]
If $Xk \in \ker \phi$, with $k \in A\cap B$, then $k \in (A \cap Y)(X \cap B)$, and so it follows that $K \leq X(A \cap Y)$ (as $K \geq X$). Conversely take, $k \in X(A \cap Y) \subseteq X(A \cap B)$ whose image is trivial. Then $\frac{K}{X} = \frac{X(A\cap Y)}{X}$ and thus $\ker \phi = \frac{X(A \cap Y)}{X}$. 
\end{proof}


\begin{proof}[Proof of Refinement Theorem]
Let the two subnormal series be:
\begin{align*}
G &= G_0 \triangleright G_1 \triangleright \cdots \triangleright G_{i-1} \triangleright G_i \triangleright \cdots \triangleright G_r = 1 &(1)\\
  &= H_0 \triangleright H_1 \triangleright \cdots \triangleright H_{j-1} \triangleright H_j \triangleright \cdots \triangleright H_s = 1 &(2)
\end{align*}
We construct a refinement of $(1)$ by inserting between $G_{i-1}$ and $G_{i}$ the following:
\[ G_{i-1} = G_i (G_{i-1} \cap H_0) \triangleright G_i(G_{i-1} \cap H_1) \triangleright \cdots \triangleright G_i (G_{i-1} \cap H_s) = G_i \]
Now the butterfly lemma implies that $G_i(G_{i-1} \cap H_{j-1} \triangleright G_i(G_{i-1} \cap H_j)$, so this is indeed a refinement of $(1)$. Similarly, construct a refinement of $H$ by inserting 
\[ H_{j-1} = H_j (H_{j-1} \cap G_0) \triangleright H_j(H_{j-1} \cap G_1) \triangleright \cdots \triangleright H_i (H_{j-1} \cap G_r) = H_j \]
and again the butterfly lemma show that this is a refinement. Both of these refinements are of length $rs$. Now we again apply the butterfly lemma to get:
\[ \frac{G_i(G_{i-1} \cap H_{j-1})}{G_i(G_{i-1} \cap H_j} \simeq \frac{H_j (G_{i-1} \cap H_{j-1})}{H_j(G_i \cap H_{j-1})} \]
for $1 \leq 1 \leq r$ and $1 \leq j \leq s$.
\end{proof}

\begin{thm}[Jordan-H\"{o}lder Theorem]
In a group $G$ with a composition series, every composition series for $G$ is isomorphic to the given one.
\end{thm}
\begin{proof}
From the refinement theorem there exists isomorphic refinements of any to subnormal series, and since composition series cannot be further refined, they must be isomorphic.
\end{proof}

\begin{Cor}
The composition factors and their multiplicities are a group invariant. 
\end{Cor}

\begin{rem}
If $G$ is a finite abelian group, then the composition factors are abelian groups and hence are cyclic of prime order.
\end{rem}

\begin{rem}
It is not the case that knowledge of the composition factors gives knowledge of the group isomorphism type. For example, $\ZZ_4 \triangleright \ZZ_2 \triangle 1$ and $\ZZ_2 \times \ZZ_2 \triangleright \ZZ_2 \triangleright 1$ have the same composition factors.
\end{rem}


\begin{Lem}
Any composition factor of a group is a simple group. 
\end{Lem}
\begin{proof}
Take $G \triangleright \cdots \triangleright G_{i-1} \triangleright G_i \triangleright \cdots \triangleright 1$, and suppose $G_{i-1} / G_i$ is not simple. Then there exists some $N / G_{i} \triangleleft G_{i-1} / G_i$ and by the correspondence theorem $G_{i-1} \triangleright N \triangleright G_i$, which contradicts the definition of a composition series.
\end{proof}

\begin{Lem}
A simple abelian group is cyclic of prime order. A composition factor of a finite abelian group is cyclic of prime order.
\end{Lem}

\begin{proof}
If $G$ is a simple abelian group, then for $1 \neq x \in G$ we have $\langle x \rangle G$ and $\langle x^2 \rangle \leq \langle x \rangle$. But $G$ is simple, so there are two cases: if $\langle x^2 \rangle  = 1$ then $G = \ZZ_2$; else $\langle x^2 \rangle = G$, so $x \in \langle x^2 \rangle$ and thus $x^{2n}$ for some $n \in \NN$. Thus $|x|$ is finite and so $G$ is finite. Take $p | |G|$. Then there exists some $H \triangleleft G$ with $|H| = p$ and so $|G| = p$.
\end{proof}

\newpage
\section{Thursday $26^{th}$ April}


\begin{rem}
Not every group has a composition series, for example $(\ZZ,+)$. There certainly exist subnormal series, but every one can be refined. 
\end{rem}



\subsection{Soluble groups}
These are also sometimes called solvable groups. 

\begin{Def}
A group $G$ has a \textbf{normal series} if the series:
\[ G = G_0 \triangleright G_1 \triangleright \cdots \triangleright G_r = 1 \]
with $G_i \triangleleft G$.
\end{Def}

\begin{eg}
$S_4 \triangleright A_4 \triangleright V \triangleright 1$ (with $V$ from the previous context) is a normal series.
\end{eg}


\begin{Def}
A group $G$ is \textbf{soluble} if it has a normal series:
\[ G = G_0 \triangleright G_1 \triangleright \cdots \triangleright G_r = 1 \]
where $G_i / G_{i+1}$ is abelian for all $1 \leq i \leq r-1$.
\end{Def}

\begin{rem}
If $G$ is abelian, then $G$ is soluble.
\end{rem}

\begin{eg}
$S_3$, which has the series $S_3 \triangleright A_3 \triangleright 1$, is soluble.
\end{eg}

\begin{rem}
Recall that if $G / N$ is abelian then $G' \leq N$. So $G_i / G_{i+1}$ then $G_i ' \leq G_{i+1}$.
\end{rem}

\begin{Def}
The derived series of $G$ is a descending series of subgroups 
\[ G = G^{(0)} \triangleright G^{(1)} \triangleright \cdots \triangleright G^{(r)} = 1 \]
Where $G^{(1)} = [G^{(0)}, G^{(0)}] = G'$ and $G^{(i+1)} = [[G^{(i)}, G^{(i)}]$. 
\end{Def}

\begin{eg}
$S_4 \triangleright A_4 \triangleright V \triangleright 1$ (with $V$ from the previous context) is the derived series of $G$. 
\end{eg}

\begin{Lem}
IF $\{ G^{(i)}$ is a derived series for $G$ then:
\begin{enumerate}
\item $G^{(i)}$ is characteristic in $G$ for all $i$;
\item $G^{(i)} / G^{(i+1)}$ is abelian; and
\item If $G_0 \triangleright G_1 \triangleright \cdots \triangleright G_n = G$ be an abelian series, then $G^{(i)} \leq G_i$ for each $i$
\end{enumerate}
\end{Lem}

\begin{proof}
\begin{enumerate}
\item Induction
\item Direct by the previous remark
\item The base case is clear, the induct on the indices. as $G^{(i+1)} = G^{(i)}{}' \leq (G_{i})' \leq G_{i+1}$ because $G_{i} / G_{i+1}$ is abelian.
\end{enumerate}
\end{proof}

\begin{Def}
If $G^{(n)} = 1$ for some $n$, then the smallest such $n$ is called the \textbf{Derived Length} of $G$. We write $\text{dl}(G) = n$. 
\end{Def}

\begin{Cor}
$G$ is soluble if and only if there exists an $n$ such that $G^{(n)} = 1$. 
\end{Cor}

\begin{Cor}
If $G$ is soluble then $\text{dl}(G)$ is the minimum length of any abelian series for $G$. 
\end{Cor}

\begin{thm}
The following are equivalent:
\begin{enumerate}
\item $G$ is soluble;
\item $G$ has a subnormal series $\{G_j\}$ such that each $G_i / G_{i+1}$ is abelian.
\item There exists an $n$ such that $G^{(n)} = 1$. 
\end{enumerate}

\end{thm}

\begin{proof}
(1) implies (2) by definition. (2) implies (3) through a simple inductive argument. (3) implies (1) as the derived series is a normal series with abelian factors. 
\end{proof}


\begin{eg}
Take $G = SL(2,3)$. Then $G \triangleright G' \triangleright G'' = Z(G) \triangleright G^{(3)} = 1$, and hence $SL(2,3)$ is a soluble group. 
\end{eg}

\begin{Ex}
Is $G = SL(2,5)$ soluble?
\end{Ex}

\begin{thm}
Subgroups and factor groups of a soluble group are themselves soluble.
\end{thm}

\begin{proof}
If $S \leq G$ then $S^{(\text{dl}(G))} \leq G^{(\text{dl}(G))} = 1$ so $S$ is soluble. \\
\\
If $N \triangleleft G$ and $\phi$ is the canonical homomorphism from $G$ to $G / N$, then $\phi(G^{(k)}) = (G / N)^{(k)}$ (which can be seen by a simple induction), and so 
\[ 1 = \phi(\{1\}) = \phi( G^{(\text{dl}(G))} ) = (G / N)^{(\text{dl}(G))} \]
and thus $G / N$ is soluble. 
\end{proof}


\begin{Lem}
Suppose $N \triangleleft G$, and that $G / N$ and $N$ are both soluble. Then $G$ is soluble and $\text{dl}(G) =  \text{dl}(G / N) + \text{dl}(N)$.
\end{Lem}

\begin{proof}
Take
\begin{align*}
\frac{G}{N} &= \frac{G_0}{N} \geq \cdots \geq \frac{G_s}{N} = \frac{N}{N} \\
N &= N_0 \geq N_1 \geq \cdots \geq N_t = 1  
\end{align*}
Then each $G_i$ contains $N$, and $G_i \triangleleft G_{i-1}$. $G_{i-1} / G_i$ is abelian because it is isomorphic to $\frac{G_{i-1}/N}{G_i / N}$ by the third isomorphism theorem. Hence
\[ G = G_0 \geq \cdots \geq G_{s-1} \geq N_0 \geq \cdots \geq N_t = 1 \] 
\end{proof}


\begin{rem}
Just because $G/N$ and $N$ are abelian, $G$ need not be abelian. Consider $G = S_3$ and $N = A_3$.
\end{rem}




\newpage
\section{Monday $30^{th}$ April}

\begin{Lem}
The direct product of a finite set of soluble groups is soluble. 
\end{Lem}

\begin{proof}
Take $G = G_1 \times G_2$. Since $G_1$ is soluble $G / G_1 \simeq G_2$ is soluble, and so $G$ is soluble. 
\end{proof}

\begin{rem}
A finite group need not be soluble even if all its subgroups are soluble. Consider $A_5$, for example.
\end{rem}


\subsection{Chief Series}
\begin{Def}
A \textbf{Chief Series} of $G$ is a series of normal subgroups
\[ G = G_0 \geq G_1 \geq \cdots \geq G_r = 1 \]
such that each factor $G_{i} / G_{i+1}$ is a minimal normal subgroup of $G / G_{i + 1}$ for $0 \leq i < r$. 
\end{Def}

\begin{rem}
This is a normal series that cannot be further refined, analogous to how a composition series is a subnormal series that cannot be further refined. 
\end{rem}

\begin{eg}
Take $S_4 \leq A_4 \leq V \leq 1$ as a chief factors of $S_4$, with chief factors $\ZZ_2, \ZZ_3,$ and $\ZZ_2 \times \ZZ_2$. 
\end{eg}

\begin{rem}
The refinement theorem and Jordan / H\"{o}lder theorem's have analogues in normal / chief series:
\begin{itemize}
\item  Any two normal series have isomorphic refinements; and
\item The chief series of $G$ is unique. 
\end{itemize}
The proofs are also analogous. 
\end{rem}




\begin{Lem}
Let $N$ be a minimal normal subgroup of a group $G$ that is finite and soluble. Then $N$ is an elementary abelian $p$-group for some prime $p$.
\end{Lem}

\begin{Ex}
If $N \triangleleft G$ and $K$ characteristic in $G$ then $K \triangleleft G$.
\end{Ex}

\begin{proof}
As $N > 1$ and $N$ is soluble, $N' < N$, and $N' \triangleleft G$. But then as $N$ is minimal, so $N' = 1$ and hence $N$ is abelian. \\
\\
Now take $p \big| |N|$ and let $A = \{ x \in N : x^p = 1 \}$. Then $1 < A < N$ as $N$ is abelian, and $A$ is characteristic in $N$, so it is normal in $G$. But then again $N$ is mininal, so we must have $A = N$, and so $N$ is a $p$-group. 
\end{proof}


\begin{Cor}
All chief factors of finite soluble groups are elementary abelian $p$-groups for various primes $p$. 
\end{Cor}

\begin{Def}
A group $G$ is \textbf{characteristically simple} if the only characteristic subgroups of $G$ are $1$ and $G$.
\end{Def}

\begin{Lem}
Every chief factor of a group $G$ is characteristically simple.
\end{Lem}

\begin{proof}
$G_i / G_{i+1} \triangleleft G / G_{i+1}$, so if $K / G_{i+1}$ is characteristic in $G_i / G_{i+1}$ then it is normal in $G / G_{i+1}$, so $K \triangleleft G$ and so either $K = G_i$ or $K = G_{i+1}$. 
\end{proof}

\begin{thm}
A finite characteristically simple group is a direct product of isomorphic simple groups.
\end{thm}
The proof is left as an exercise. We have already done this for finite simple $G$.  


\begin{thm}
Let $M$ be a proper maximal subgroup of a finite soluble group $G$.  Then $|G:M|$ is a prime power.
\end{thm}

\begin{proof}
Let $L$ be maximal among normal subgroups of $G$ contained in $M$, i.e. $L = \text{Core}_G(M)$. Since $L \triangleleft G$, take $K / L$ as a chief factor of $G$ (Take $K$ such that $K / L$ is minimal in $G / L$). Then $K > L$ and $K \triangleleft G$, so $K$ is not contained in $M$. Thus $KM > M$ and so $KM = G$. By the second isomorphism theorem, $|G : M| = |K : K \cap M| \big| |K : L|$. But $K  / L$ is a chief factor, and hence is an elementary abelian group and so $|K : L |$ is a prime power, and so $|G:M|$ is too.
\end{proof}

\begin{rem}
The last result does not hold without the soluble hypothesis, as otherwise we cannot claim $K / L$ is a chief factor. Consider again $A_5$ with 3 conjugacy classes of maximal subgroups with indices $5,6,10$. Also $PSL(2,7) = SL(2,7) / \ZZ$ has maximal subgroups of indices $7$ and $8$.
\end{rem}


\newpage
\section{Tuesday $1^{st}$ May}
\subsection*{Nilpotent Groups}

Recall that a finite group $G$ is called nilpotent if all its Sylow $p$-subgroups are normal in $G$. In this case:
\[ G = \prod_{p \in \pi} S_p \qquad \text{where} \ \pi = \{ \, \text{primes dividing} \ |G| \} \]
We now generalise.
\begin{Def}
A group $G$ is nilotent if $G$ has a finite series
\[ G = G_o > G_1 > \cdots > G_r = 1 \]
where $G_i / G_{i+1} \leq Z(G / G_{i+1})$ for $0 \leq i < r$.
\end{Def}

\begin{rem}
Such a series is a normal series, as $G_i / G_{i+1} \leq Z(G / G_{i+1})$ implies $G_i / G_{i+1} \triangleleft G / G_{i+1}$ and so $G_i \triangleleft G$. 
\end{rem}

\begin{eg}
Abelian groups are nilpotent, which can be seen by taking the series $G = G_0 \triangleright G_1 = 1$. 
\end{eg}

\begin{eg}
Consider $G = D_4 = \langle (1234), (14)(23) \rangle$. Then the series $D_4 > Z(D_4) = \langle (13)(24) \rangle > 1$ show that $D_4$ is nilpotent. 
\end{eg}

\begin{eg}
The series
\[ D_8 > \langle (1753)(2864) \rangle > \langle (15)(26)(37)(48) \rangle > 1 \]
show $D_8$ is nilpotent.
\end{eg}

\begin{eg}
We can show $S_3$ is not nilpotent as this would imply $A_3$ is in the center of $S_3$, but this is just the trivial subgroup. In general nilpotent groups cannot have a trivial centre. 
\end{eg}

\begin{rem}
Nilpotency implies solubility.  
\end{rem}

\begin{Lem}
Let $G = G_0 \triangleright \cdots \triangleright G_r = 1$ be a normal series for $G$. It is a central series if and only if $[G_i, G] \leq G_{i+1}$ for $0 \leq i < r$.
\end{Lem}

\begin{proof}
$G_i / G_{i+1} \leq Z(G / G_{i+1})$ if and only if $[xG_{i+1}, yG_{i+1}] = G_{i+1}$ for $x \in G_i$ and $y \in G$. But this is equivalent to $[x,y] G_{i+1} = G_{i+1}$, or $[x,y] \in G_{i+1}$. Hence $[G_{i}, G] \leq G_{i+1}$. 
\end{proof}


\begin{Lem}
Subgroups and factors groups of nilpotent groups are nilpotent. 
\end{Lem}

\begin{proof}
Take a central series $G = G_0 \triangleright G_1 \triangleright \cdots \triangleright G_r = 1$. Let $S \leq G$. Then the series
\[ S = S \cap G_0 \geq S_1 = S \cap G_1 \geq \cdots \geq S \cap G_i \geq \cdots \geq S \cap G_r = 1 \]
has $S_i = S \cap G_i \triangleleft S$. and $[S_i, S] \leq S_{i+1}$, so this is a central series of $S$. Now take $N \triangleleft G$. Then consider the series
\[ \frac{G}{N} = \frac{G_0 N}{N} \geq \frac{G_1 N }{N} \geq \cdots \frac{G_r N}{N} = N \]
For each $i$, $G_iN / N \trianglerighteq G_{i+1}N / N$, so for $x \in G$ and $y \in G_{i-1}$ we have $[yN, xN] = [y,x]N \in G_i N / N$. Thus $[ G_{i-1}N / N, G / N ] \leq G_{i}N / N$. 
\end{proof}

\begin{Lem}
The direct product of a finite set of nilpotent groups is nilpotent
\end{Lem}

\begin{proof}
Take nilpotent groups $H, K$ with respective central series:
\begin{align*}
H &= H_0 \geq \cdots \geq H_r = 1 \\
K &= K_0 \geq \cdots \geq K_r = 1
\end{align*}
Then the series
\[ H_0 \times K_0 \geq H_1 \times K_1 \geq \cdots \geq H_r \times K_r = 1 \]
is a central series for $H \times K$. It is easy to show that $G_i = H_i \times K_i \triangleleft G = H \times K$. We can also see that $[G_i, G] \leq G_{i+1}$ as
\[ H_i \times K_i, H \times K] = [H_i, H] \times [K_i, K] \leq H_{i+1} \times K_{i+1} = G_{i+1} \]
\end{proof}

\begin{rem}
The direct product of an arbitrary set of nilpotent groups is not necessarily nilpotent. 
\end{rem}

\begin{Def}
The \textbf{Nilpotency Class} of a group $G$ is the length of the shortest central series for $G$. If $G$ is nilpotent then $cl(G) < c$ for some $c \in \NN$. 
\end{Def}

\begin{eg}
Consider the collection of groups $G_i$ where each $G_i$ has $cl(G_i) = i$, then the product:
\[ G = \prod_{n \in \NN} G_n \]
is not nilpotent as its nilpotency class is infinite.
\end{eg}

\begin{Lem}
Let $G$ be a finite group with $Z(G / M) > 1$ for every proper $M \triangleleft G$. Then $G$ is nilpotent.
\end{Lem}
\begin{proof}
Define a series $Z_0 = 1$, and $Z_1 = Z(G)$. The define $Z_i \leq G$ such that $Z_i / Z_{i-1} = Z(G / Z_{i-1})$. By our hypothesis if $Z_i < G$ then $Z_{i+1} > Z_i$. Since $G$ is finite, the $Z_n = G$ for some $n \in \NN$ (as we run out of elements eventually). This is the clearly a central series (with appropriate relabelling) and so $G$ is nilpotent. 
\end{proof}



\begin{Cor}
A finite $p$-group is nilpotent
\end{Cor}
\begin{proof}
$Z(P)$ is nontrivial for any finite $P$-group $P$. 
\end{proof}







\newpage
\section{Thursday $3^{rd}$ May}

\begin{Lem}[Frattini Lemma]
Let $ N \triangleleft G$ with $N$ finite, and let $P \in \text{Syl}_p(N)$. Then $G = N_G(P) N$.
\end{Lem}
\begin{proof}
Let $g \in G$, and consider $P^g \subseteq N^g = N$. Since $|P^g| = |P|$, $P^g \in \text{Syl}_p(N)$. Thus there exists some $n \in N$ such that $P^{gn} = P$.  Thus $gn \in N_G(P)$, or equivalently $g \in N_G(P) n^{-1} \subseteq N_G(P)N$. Since $g$ was arbitrary, we then get $G =  N_G(P)N$.
\end{proof}


\begin{thm}
Let $G$ be finite. Then the following are equivalent:

\begin{enumerate}
\item $G$ is nilpotent;
\item $N_G(H) > H$ if $H < G$; 
\item Every maximal subgroup of $G$ is normal in $G$;
\item Every Sylow $p$-subgroup is normal in $G$; and
\item $G$ is isomorphic to a direct product of $p$-groups for all $p \big| |G|$.
\end{enumerate}

\end{thm}

\begin{rem}
(1) - (4) are equivalent when $G$ is infinite, but 5 does not hold. 
\end{rem}

\begin{proof}
If $G$ is nilpotent, take the central series $G_0 \geq G_1 \geq \cdots G_r = 1$. Take $H \leq G$ and suppose $G_k \leq H$ for some $k > 0$ (with $k$ minimal, so $G_{k-1} \not\leq H$). Then $[G_{k-1}, G] \leq G_k \leq H$, so $[G_{k-1} , H] \leq H$, and so $G_{k-1}$ normalises $H$. But $G_{k-1} > H$, so we get (1) implies (2). \\
\\
Now let $M < G$ be a maximal subgroup. Then $N_G(M) > M$ (by (2)) so $N_G(M) = G$, and so $M \triangleleft G$. Thus (2) implies (3).\\
\\
Let $P \in \text{Syl}_p(G)$. If $N_G(P) < G$, then choose a maximal subgroup $M < G$ with $N_G(P) \leq M$, so that $P \in \text{Syl}_p(M)$. By (3), $M \triangleleft G$, so by the Frattini Lemma $G = N_G(P) M$. But $M \geq N_G(P)$ so $N_G(P) M = M$, a contradiction. Thus $N_G(P) = G$ and so $P \triangleleft G$. \\
\\
We have already proved that (4) implies (5) and (5) implies (1), so we are done.
\end{proof}

\begin{rem}
If $M$ is a maximal subgroup of $G$, a finite soluble group, then $|G : M|$ is a prime power.\\
\\
If $M < G$ has $|G : M| = p$ then $M$ is a maximal subgroup of $G$, however the converse of this does not hold (consider $S_4$
\end{rem}

\begin{Lem}
Let $G$ be a nilpotent group and let $M$ be a maximal subgroup of $G$. Then $G / M$ has prime order. 
\end{Lem}

\begin{proof}
This follows from the previous lemma, a the fact that $G / M$ can have no proper subgroup (by the third isomorphism theorem).
\end{proof}

\begin{rem}
A nilpotent group need not have any maximal subgroups. Consider $\ZZ_p^\infty = \{ z \in \CC : z^{p^n} = 1 \}$ where $n \in \mathbb{P}$ under multiplication in $\CC$ (this is called a Pr\"{u}fer group), a countable abelian group. Then $H_n = \left( \frac{1}{p^n} \right) \ZZ / \ZZ$, the cyclic subgroup of $\ZZ_p^\infty$ with $p^n$ elements (those with order dividing $p^n$.  Then $H_i \subseteq H_{i+1} \subseteq \ZZ_p^\infty$ for all $i \in \NN$, so $\ZZ_p^\infty$ has no maximal subgroup. 
\end{rem}


\subsection{General Central Series}

\begin{Def}
Take a group $G$ and define $\gamma_1 (G) = G$. Then let $\gamma_i (G) = [\gamma_{i-1} (G), G]$. If $\gamma_{c+1} = 1$ for some $c \geq 0$, then $G$ has a \textbf{Lower Central Series} given by
\[ G = \gamma_1(G) \geq \cdots \geq \gamma_{c+1}(G) = 1 \]
\end{Def}

\begin{Lem}
A lower central series for $G$ is a central series
\end{Lem}

\begin{proof}
excercise
\end{proof}

\begin{Def}
Take a group $G$ and let $Z_0(G) = 1$, then define $Z_i(G)$ by $Z_i(G) / Z_{i+1}(G) = Z( G / Z_{i-1}(G) )$. If $Z_r(G) = G$ for some $r \geq 0$, then $G$ has \textbf{Upper Central Series} 
\[ G = Z_r(G) \geq \cdots \geq Z_0 = 1 \]
\end{Def}

\begin{Lem}
An upper central series is a central series
\end{Lem}

\begin{proof}
Trivial.
\end{proof}


\begin{Ex}
Find the upper and lower central series of $D_4$. 
\end{Ex}

\begin{eg}
$D_4 \leq \gamma_1(D_4) = D_4 > \gamma_2 = G' = \langle (13)(24) \rangle > \gamma_3 = [\gamma_2 , G] = 1$
\end{eg}





\newpage
\section{Monday $7^{th}$ May}

\begin{eg}
The two composition series of $Q_8 = \langle a, b\rangle = \langle (1625)(3847), (1423)(4768) \rangle$ are
\[ Q_8 > \langle a \rangle > \langle a^2 \rangle > 1 \qquad \text{and} \qquad Q_8 > \langle b \rangle > \langle b^2 \rangle > 1 \]
The upper and lower central series for this group are both:
\[ Q_8 > Q_8' = \langle (12)(34)(56)(78) \rangle > 1 \]
\end{eg}

\begin{thm}
If $G$ has a central series $G = G_0 \geq G_1 \geq \cdots \geq G_r = 1$. Then:
\begin{itemize}
\item $G_{r - i} \leq Z_i$ for $0 \leq i \leq r$; and
\item $G_i \geq \gamma_{i + 1}$ for $0 \leq i \leq r-1$. 
\end{itemize}
\end{thm}
\begin{rem}
Colloquially, this means that the upper central series goes up as fast as any others, whilst the lower central series goes down as fast as any others. 
\end{rem}
\begin{proof}
Left as an exercise, use induction.
\end{proof}


\begin{Cor}
If a group is nilpotent, then its Upper central series and lower central series have the same length. 
\end{Cor}

\begin{proof}
Assume $G$ has nilpotency class $r$. Then it has a central series of length $r$. This central series is at least as long as the Upper Central Series and Lower Central Series. But both of these are central series, so they must have the same length. 
\end{proof}


\begin{eg}
Consider $g = Q_8 \times \ZZ_2 = \langle a,b,c : a^2 = b^2 = (ab)^2, c^2 = 1 \rangle$. We see that
\[ \gamma_1 = G > \gamma_2 = \langle a^2 \rangle > \gamma_3 = 1 \]
whilst 
\[ Z_0 = 1 < Z_1 = Z(G) = \langle a^2, c \rangle < Z_2(G) = G \]
\end{eg}


\subsection{Minimal Normal Subgroup}

We have seen that if $G$ is a soluble group, then its minimal normal subgroup is an elementary abelian $p$-group. 

\begin{Lem}
Let $N$ be a non-trivial normal subgroup of a finite nilpotent group $G$. Then $N \cap Z(G) > 1$. 
\end{Lem}

\begin{proof}
$G$ is nilpotent, so $G$ has an upper central series $Z_0 < \cdots < Z_{m-1} < Z_m < \cdots < Z_c$. Let $N \triangleleft G$. Then exists a minimal $m$ such that $N \cap Z_m \neq 1$. Then consider $[N \cap Z_m, G] \leq N \cap [Z_m, G] \leq N \cap Z_{m-1} = 1$. Thus $N \cap Z_m \neq 1$ is central in $G$ and so $N \cap Z(G) > 1$.  
\end{proof}

\begin{Cor}
A minimal normal subgroup of a finite nilpotent group is of prime order. 
\end{Cor}

\begin{proof}
$N < Z(G)$,  so $N$ has prime order.
\end{proof}



\subsection{Finite $p$-Groups}

\begin{Lem}
A finite simple $p$-group $P$ must have order $p$
\end{Lem}

\begin{proof}
$1 < Z(P) \triangleleft P$ so $Z(P) = P$. Thus $P$ is abelian and so $|P| = p$.
\end{proof}

\begin{Lem}
Let $P$ be a finite $p$-group. Then every composition factor and chief factor of $P$ has order $p$. 
\end{Lem}

\begin{proof}
$P$ is nilpotent, take its UCS. Refine until you have an abelian composition series. 
\end{proof}

\begin{Lem}
Take a finite non-trivial $p$-group $P$. $P$ has a subgroup of index $p$ and every such subgroup is normal.
\end{Lem}

\begin{proof}
$P > 1$, so choose a maximal normal subgroup $N \triangleleft P$. $P / N$ is simple, so $|P : N| = p$. Then $N_P(G) > N$ and so $N \triangleleft P$. 
\end{proof}


\begin{Lem}
Every maximal subgroup of $P$ is normal and has index $p$.
\end{Lem}

\begin{proof}
Let $H < P$ be a maximal subgroup of $P$. Then $N_P(H) > H$ so $H \triangleleft P$. Also, since $H$ is maximal, $P / H$ has no nontrivial subgroups, and so $P / H$ has prime order. 
\end{proof}

\begin{Lem}
Let $P$ be a finite $p$-group. Let $N < M$ be normal subgroups of $P$. Then there exists an $L \triangleleft P$ such that $N \leq L \leq M$ and $|L : N| = p$. 
\end{Lem}

\begin{proof}
Let $\overline{P} = P / N$. $\overline{M}$ is a nontrivial subgroup and $\overline{M} \triangleleft \overline{P}$. Observe that $Z(\overline{P}) \cap \overline{M} \neq 1$ as $\overline{P}$ is nilpotent. Hence this intersection contains an element $x$ of order $p$. Then set $\overline{L} = \langle x \rangle \triangleleft \overline{P}$, so pulling back via the correspondence theorem gives $N \leq L \leq M$, $L \triangleleft P$, and $|L:N| = p$.
\end{proof}



\newpage
\section{Tuesday $8^{th}$ May}

\subsection{Finite $p$-groups}

\begin{Cor}
Given a finite abelian $p$-group $P$ with $|P| = p^n$, for every $b \in \{0, \dots, n\}$ there exists an $L_b \triangleleft P$ with $|L_b| = p^b$. 
\end{Cor}

\begin{proof}
This is trivially true for $b=0$. If it is true for some $b$, then apply the previous lemma with $N = L_b$ and $M = P$ to get $L_{b+1}$ with $N \leq L_{b+1} \leq M$ with $L_{b+1} \triangleleft P$ and $|L_{b+1} : L_b| = p$. Hence $|L_{b+1}| = p^{b+1}$, so by induction we are done. 
\end{proof}

\begin{Cor}
With $G$ be a finite group with $p^b | |G|$, ($b \in \NN$ and $p$ prime). Then $G$ has a subgroup of order $p^b$. 
\end{Cor}
This is a stronger version of the Sylow existence theorem. 


\subsection{Largest Nilpotent Subgroup}

Let $G$ be a finite group and $S \in \text{Syl}_p(G)$. Consider $\text{Core}_G(S) = \cap_{g \in G} S^g = \cap_{T \in \text{Syl}_G(p)} T$. Let $N \triangleleft G$, so that $S \cap N \in \text{Syl}_p(N)$. In particular, if $N$ is a $p$-subgroup of $G$, then $N \leq S$. Hence $\text{Core}_G(S)$ contains every normal $p$-subgroup of $G$. 

\begin{Def}
We define $o_p(G) = \text{Core}_G(S) = \cap \text{Syl}_p(G)$. 
\end{Def}

\begin{rem}
$o_p(G)$ contains every normal $p$-subgroups of $G$, and it is the largest normal $p$-subgroup of $G$. 
\end{rem}

\begin{Lem}
$o_p(G)$ is characteristic in $G$.
\end{Lem}

By taking direct products of $o_p(G)$ for each $p \big| \, |G|$ gives the largest nilpotent subgroup.

\subsection{Frattini Subgroup}

\begin{Def}
Let $G$ be a finite group. Define the \textbf{Frattini Subgroup} of $G$ as
\[ \Phi(G) = \bigcap_{M < \cdot G} M \]
i.e., the intersection of all maximal subgroups 
\end{Def}

\begin{Ex}
$\Phi(G)$ is characteristic in $G$
\end{Ex}

\begin{Lem}
The following are equivalent:
\begin{enumerate}
\item $G$ is nilpotent;
\item $G / \Phi(G)$ is abelian; and
\item $G / \Phi(G)$ is nilpotent.
\end{enumerate}
\end{Lem}

\begin{proof}
Suppose $G$ is niloptent, any maximal subgroup $M$ is normal in $G$ and $G / M$ has prime order and is hence abelian. Thus $G' \leq M$ and so $G' \leq \Phi(G)$, giving us an abelian $G / \Phi(G)$. \\
\\
(2) implies (3) trivially; abelian groups are nilpotent. Now suppose $G / \Phi(G)$ is nilpotent. Let $M$ be any maximal subgroup, which must then contain $\Phi(G)$. $M / \Phi(G)$ is maximal in $G / \Phi(G)$, which is nilpotent, so $M / \Phi(G) \triangleleft G / \Phi(G)$. Thus $M \triangleleft G$ and so $G$ is nilpotent. 
\end{proof}

\begin{Lem}
Let $P$ be a finite $p$-group. Then $P / \Phi(P)$ is a elementary abelian $p$-group. 
\end{Lem}

\begin{proof}
If $M < \cdot P$ then $P / M$ is cyclic of prime order, so $P' \leq M$ and $x^p \in M$ for all $x \in P$. Thus $P' \leq \Phi(P)$ and $x^p \in \Phi(P)$ for all $x \in P$, giving us the result. 
\end{proof}

\begin{rem}
If $P$ is a finite $p$-group, then byt Burnside basis theorem says that $P$ needs precisely $d$ elements to generate it (where $|P : \Phi(P)| = p^d$. \\
\\
$\Phi(P)$ consists of non-generators; elements that can always be removed from generating sets without changing the structure.
\end{rem}

\subsection{Simplicity of $A_n$ for $n \geq 5$}

\begin{rem}
$A_5$ is not abelian, as we can find two elements which do not commute (exercise). \\
\\
If $N \triangleleft G$ then $N$ is a union of conjugacy classes of $G$, so if $G$ is simple then it is a complete union of conjugacy classes.
\end{rem}

\begin{prop}
$A_5$ is simple.
\end{prop}

\begin{proof}
Consider the size of the conjugacy classes of $1$, $(12)(34)$, $(123)$, $(12345)$ and $(13452)$: respectively, $1$, 15, 20, 12, and 12.
So if $N \triangleleft A_5$ then $|N| = 1$ or $|N| = 60$. 
\end{proof}

\begin{Lem}
Let $n \geq 3$. Every element of $A_5$ can be written as a product of $3$-cycles.
\end{Lem}

\begin{proof}
Every element of $A_n$ is a product of transpositions. Take two transpositions $h_1 = (ab)$ and $h_2 = (cd)$. If $h_1 = h_2$ then $h_!h_2 = 1$. If they share one element (WLOG $b=c$) then $h_1h_2 = (adb)$. Otherwise $h_1h_2 = (abc)(adc)$ Thus every element of $A_5$ is a product of three cycles. 
\end{proof}

\begin{Lem}
$Z(S_n) = 1$ for all $n \geq 3$.
\end{Lem}


\newpage
\section{Thursday $10^{th}$ May}


\begin{Lem}
Let $G$ act transitively on $\Omega$. Then $\{ G_\alpha : \alpha \in \Omega \}$ is a single conjugacy class of subgroups. Each $G_\alpha$ is called a one-point stabiliser. 
\end{Lem}

\begin{proof}
Let $\alpha \in \Omega$ and $g \in G$. Let $\beta = \alpha \cdot g$, then if $x \in G$ fixes $\alpha$ then
\[ \beta \cdot x^g = (\alpha \cdot g) \cdot x^g = \alpha \cdot (xg) = \alpha \cdot g = \beta\]
i.e., $x^g$ fixes $\beta$. Thus every conjugate of $G_\alpha$ is another one-point-stabiliser. Also, if $\alpha, \beta \in \Omega$, then by the transitivity of the action of $G$, there exists a $g \in G$ such that $\alpha \cdot g = \beta$ and $(G_\alpha )^g = G_\beta$. Hence every two one-point-stabilisers are conjugate. 
\end{proof}

\begin{Lem}
$A_5$ is simple.
\end{Lem}


\begin{proof}
Let $N \triangleleft G = A_5$. Suppose $3\big| |N|$. Then $N$ contains a Sylow $3$-subgroup. Since $N \triangleleft G$, it contains all Sylow $3$-subgroups, which collectively contain $20$ elements of order 3, and so $|N|$ must exceed $20$. Similarly, if $5 \big| |N|$, then $|N| > 24$ and so must be $30$. But then it must contain 20 elements of order 3 as well, and so would have to have at least 44 elements: a contradiction. Hence neither 3 nor 5 divides $|N|$. So $|N|$ is either $2$ or 4. If it were 4, it would be the unique sylow 2-subgroup and would contain 15 elements of order 2: another contradiction. Lastly, if $|N| = 2$, then take $x \in N$, which must have some fixed point. So $N = \{ 1 , x\} \leq G_\alpha$ where $\alpha \in \{1, \dots, 5\} = \Omega$. But then $G_\beta = (G_\alpha)^y$ so $N \leq G_\beta$ for all $\beta \in \Omega$. But $x$ cannot fix every point, another contradiction, so there are no options for non-trivial normal subgroups of $A_5$.
\end{proof}


\begin{Lem}
The only normal subgroup of $A_n$ that contains a 3-cycle is $A_n$.
\end{Lem}


\begin{proof}
Let $N \triangleleft A_n$. This is easy to check for $n =3$ and $n=4$. Take $n \geq 5$. Let $\pi$ be the $3$-cycle in $N$. Recall that all $3$-cycles are conjugate in $S_n$. Now there exists an odd permutation which centralises $\pi$ (seen as the support of $\pi$ contains three points, and so there are at least 2 point remaining that can be transposed, and that transposition fixes $\pi$). Thus $C_{S_n}(\pi)  > C_{A_n}(\pi) = C_{S_n}(\pi) \cap A_n$, and $A_n C_{S_n}(\pi) = S_n$. \\
\\
Then $| C_{S_n}(\pi) : C_{A_n}(\pi) | = |C_{S_n}(\pi) : C_{S_n}(\pi) \cap A_n | = | A_n C_{S_n}(\pi) : A_n | = |S_n : A_n| = 2$, so $|A_n : C_{A_n}(\pi)| = |S_n : C_{S_n}(\pi)|$. Hence all three cycles are contained in $A_n$, and so all three cycles are contained in $N$, which then implies that $N = A_n$. 
\end{proof}


\begin{thm}
$A_n$ is simple for $n \geq 5$. 
\end{thm}

\begin{proof}
We have already shown this for $n = 5$. Take $n \geq 6$, and suppose that $A_{n-1}$ is simple. Take $N \triangleleft A_n$. $N$ contains an element which fixes some $i \in \{1, \dots, n\}$. Suppose every non-identity element of $N$ has no fixed points. Let $\pi \in N$ and let $a = \pi(1) \neq 1$. Choose $b,c$ such that $c = \pi(b) \neq b$ and $b,c \neq 1,a$. Since $n \geq 6$, there exist $d,e \notin \{ 1,a,b,c \}$. Let $\rho = (1a)(bcde) \in A_n$. Then $\rho^{-1} = (1a)(bedc)$. Since $N$ is normal in $A_n$, $\rho \pi \rho^{-1} \in N$, and so $\rho \pi \rho^{-1} (a) = 1$ and $\rho \pi  \rho^{-1} (c) = d$. Thus $\rho \pi \rho^{-1} \pi (1) = 1$ and $\rho \pi \rho^{-1} \pi (b) = d$. Thus $\rho \pi \rho^{-1} \pi$ is a non-trivial element of $N$ which fixes $1$. \\
\\
Let $B \leq A_n$ consist of all permutations which fix $i$, so $B \simeq A_{n-1}$. Let $N_i  = N \cap B \triangleleft B$. But by the inductive hypothesis, $A_{n-1} = B$ is simple, so $N_i = 1$ or $B$. But $N_i$ is nontrivial (as it contains $\rho \pi \rho^{-1} \pi$, and so $N_i = B$; or equivalently, $B \leq N$. Since $n \geq 6$, $B$ contains a $3$-cycle, so $N$ contains a three cycle, and by our previous lemma must therefor contain all three cycles. Thus $N = A_n$. 
\end{proof}



\newpage
\section{Monday $14^{th}$ May}


\begin{proof}[Alternative Proof for $n > 5$]
Take $N \triangleleft A_n = G$, assume $A_{n-1}$ is simple. Let $H = G_\alpha$ for some $\alpha \in \{1, \dots, n\}$, so $H = A_{n-1}$ is simple. Then $N \cap H \triangleleft H$, so there are two cases. If $N \cap H = H$, then $H \leq N$. Now $G_\alpha \sim G_\beta$ and so all one point stabilsers are in $N$. Thus $N$ contains every element of $A_n$ that fixes a point, so it must contain every product of two transpositions, and so $N \geq A_n$. \\
\\
Otherwise $N \cap H = 1$, so the only element of $N$ that fixes a point is $1$. If $N > 1$, then take some non trivial $x \in N$. Either $x$ contains an $m$-cycle for some $m \geq 3$ or it consists entirely of transpositions. i.e., $x = (12)(34)\dots$ or $x = (12\dots)(\dots)$. Let $y = x^{(356)} \in N$. Then $y = (12)(54)\dots$ or $y = (125\dots)(\dots)$. Thus $xy^{-1}$ is a non trivial element of $N$ that fixes $1$, a contradiction. Thus $N  = 1$. 
\end{proof}



\begin{Cor}
For $n \geq 5$, the only normal subgroups of $S_n$ are $1, A_n$ and $S_n$. 
\end{Cor}
\begin{proof}
Let $N \triangleleft S_n$. Then $N \cap A_n \triangleleft A_n$, so by the simplicity of $A_n$ either $N \cap A_n = A_n$ or $1$. In the first case, we must have $A_n N = A_n$ or $S_n$, in which case we are done (TODO: Check). Otherwise take $N \cap A_n = 1$. Then $|N| \leq |S_n : A_n| = 2$. if $|N| = 2$ then $N \leq Z(S_n) = 1$, a contradiction, so $N = 1$. 
\end{proof}


\begin{rem}
$A_n$ is perfect, i.e., $A_n = A_n'$. Thus $S_n$ is insoluble.
\end{rem}

\begin{Ex}
Show that the only subgroup of $S_n$ of index less than $n$ is $A_n$.
\end{Ex}

\subsection{Classification of Finite Simple Groups}

\begin{thm}
The finite simple groups are:
\begin{enumerate}
\item Cyclic of prime order;
\item $A_n$ for $n \geq 5$; 
\item 16 families of groups; and
\item 26 sporadic groups.
\end{enumerate}
\end{thm}

\begin{eg}
$PSL(n,q) = SL(n,q) / Z(GL(n,q))$ except $(n,q) \in  \{(2,2), (2,3)\}$. Other families include $PSp(n,q)$, $U(n,q)$, and $\Omega(n,q)$ - sets of matricies that preserve a particular forms. There are also the exceptional groups $E_6(q), E_7(q)$, and $E_8(q)$. 
\end{eg}

\begin{eg}
Some of the sporadic groups are $M_{11}$, a useful group in graph theory. $\mathbb{M}$, the monster group, has $\sim 10^{53}$ elements, and contains many of the other sporadic groups, such as $C_\mathbb{M}(x)$. 
\end{eg}

\begin{thm}[Feit - Thompson Theorem]
Every finite group of odd order is soluble (and therefor not simple).
\end{thm}



\subsection{Group Constructions}

We have already seen that the direct product allows us to form new groups, and if $G = H \times K$ then the following are equivalent:\begin{enumerate}
\item $H, K \triangleleft G$;
\item $H \cap K = 1$; and
\item $KH = G$. 
\end{enumerate}

\begin{Def}
We define $G$ as the semi direct product of $H$ and $K$ if
\begin{enumerate}
\item $H \triangleleft G$;
\item $H \cap K = 1$; and
\item $KH = G$. 
\end{enumerate}
And write $G = H \rtimes K$, and so $G$ is a semidirect product \textbf{of} $H$ \textbf{by} $K$. We might also say $K$ is the complement of $H$ in $G$ if (2) and (3) hold. $G$ is the \textbf{Split Extension} of $N$ by $H$.
\end{Def}

\begin{rem}
If $G = N \rtimes H$ then
\[ \frac{G}{N} = \frac{NH}{N} \simeq \frac{H}{H \cap N} = \frac{H}{1} \]
so $G / N \simeq H$
\end{rem}


\begin{eg}
Take $D_4 = \langle x,y : x^4 = y^2 = 1, x^y = x^{-1} \rangle$. Then $N = \langle x \rangle  \triangleleft D_4$ and $H = \langle y \rangle$ has $N \cap H = 1$, and $NH = D_4$, so $D_4 = N \rtimes H$.
\end{eg}

\begin{eg}
Take $S_3 = \langle a,b : a^3 = b^2 = 1, a^b = a^{-1} \rangle$. Then set $N = \langle a \rangle$ and $H = \langle b \rangle$. Then $S_3 = N \rtimes H$. 
\end{eg}

\begin{eg}
$\ZZ_6 = \ZZ_3 \times \ZZ_2 = \langle a,b : a^3, b^2, a^b = a \rangle$. 
\end{eg}

\begin{rem}
Ever direct product (DP) is a semi direct product (SDP).
\end{rem}

\begin{eg}
$D_n = \ZZ_n \rtimes \ZZ_2$ and $S_n = A_n \rtimes \ZZ_2$. 
\end{eg}

\begin{eg}
$Q_8$ cannot be a split extension of any $N$, $H$ with $|N| = 4$, as $N$ would then have to contain the unique element of order 2.
\end{eg}





\newpage
\section{Tuesday $15^{th}$ May}

\begin{Lem}
Let $G = N \rtimes H$. For each $h \in H$, define $\theta_h : N \rightarrow N$ by $n \mapsto n^h = h n h^{-1}$. Then $\theta_h \in \text{Aut}(N)$. The map $\theta : H \rightarrow \text{Aut}(N)$ given by $h \mapsto \theta_h$ is a homomorphism. 
\end{Lem}

\begin{proof}
For the first claim, see that $N \triangleleft G$ so $n^h \in N$. $\theta_h$ is indeed Homomorpic, injective, and surjective (as $\theta_h(h^{-1} n h) = n$). \\
\\
Then $\theta(h_1 h_2) (n) = (h_1 h_2) n (h_1 h_2)^{-1} = \theta_{h_1} \circ \theta_{h_2} (n) = \theta(h_1) \circ \theta(h_2) (n)$, so $\theta$ is indeed a homomorphism.
\end{proof}



\begin{eg}
Take $N = \ZZ_3$ and $H = \ZZ_2 \simeq \text{Aut}(N)$. Then both $\ZZ_6$ and $S_3$ are semi direct products of $N$ by $H$. 
\end{eg}


\subsection{External Semi Direct Products}

\begin{Def}
$N \triangleleft G $, $H \leq G$, $\theta : H \rightarrow \text{Aut}(N)$ given by $h \mapsto \theta_h$. Then we say $G = N \rtimes H$ \textbf{Realises} $\theta$ if for all $h \in H$ and $n \in N$, then $\theta_h (n) = n^h$. 
\end{Def}

\begin{eg}
If $\theta$ is the trivial map, then $G$ is the direect product of $N$ and $H$. 
\end{eg}

\begin{Lem}
If $N \triangleleft G$, and $H$ is complement to $N$ in $G$, then every $g \in G$ has a unique expression $g = nh$ with $n \in N$ and $h \in H$.
\end{Lem}

\begin{proof}
Take $g = G$ and suppose that it has expressions $g = n_1 h_1 = n_2 h_2$. Then $n^{-1} n_2 = h_1 h_2^{-1} = 1$ (as $N \cap H = 1$), so $n_1 = n_2$ and $h_1 = h_2$. 
\end{proof}


\begin{Def}
Given $N$, $H$, $\theta: H \rightarrow \text{Aut}(N)$ with $h \mapsto \theta_h = (n \mapsto h n h^{-1})$, we define $G = N \rtimes_{\theta} H$ to be the set of ordered pairs
\[ \{ (n,h) : n \in N, h \in H \} \]
equipped with the opperation:
\[ (n_1, h_1) \cdot (n_2, h_2) = ( n_1 \theta_{h_1} (n_2), h_1, h_2 ) \]
\end{Def}

\begin{rem}
If $\theta$ maps $h \mapsto 1$ for all $h \in H$, then $N \rtimes_{theta} H = N \times H$. 
\end{rem}

\begin{thm}
Let $N$ and $H$ be groups, and let $\theta : H \rightarrow \text{Aut}(N)$. Set $G = \{ (n,h) : n \in N, h \in H \}$ , $N_0 = \{ (n,1) : n \in N \}$, and $H_0 = \{ (1, h) : h \in H \}$. Then:
\begin{enumerate}
\item $G$ equiped with the product in the previous definition is a group
\item $N_0 \triangleleft G$ and $N_0 \simeq N$; 
\item $H_0 \leq G$ and $H_0 \simeq H$; and
\item $G = N_0 \rtimes H_0$. 
\end{enumerate}

\end{thm}

\begin{proof}
$G$ is closed under the given opperation. It is associative, as
\[ (a,x) ((b,y) (c,z)) = (a \theta_x (b) \theta_y (c), xyz) = ((a,x)(b,y))(c,z))\]
Then it has identity as
\[ (n,h) (1,1) = (n \theta_h(1), h) = (n,h) \]
And it has an inverse as
\[ (n,h)(\theta_{h^{-1}}(n^{-1}), h^{-1}) = (n \theta_h \theta_{h^{-1}} (n^{-1}), hh^{-1} ) = (1,1) \]
Now $(1,h_1) (1,h_2) = (1, h_1h_2) \in H_0$ so $H_0 \leq G$. $(n_1, 1)(n_2, 1) = (n_1n_2, 1) \in N$ so $N_0 \leq G$. Then take any $(n,h) \in G$. Then $(n,h) = (n,1)(1,h)$ so $G = N_0 H_0$, and it is clear that $N_0 \cap H_0 = 1$. Lastly, see that $N_0 \triangleleft G$ as
\[ (m,h) (n,1) (m,h)^{-1} = \cdots = (n_0, 1) \in N_0 \]

\end{proof}


\begin{rem}
Given $G = N \rtimes H$, there exists a $\theta : H \rightarrow \text{Aut}(N)$ with $h \mapsto \theta_h$. If $G = N \rtimes H$, then $g = nh$, so
\[ g_1 g_2 = n_1 h_1 n_2 h_2 = n_1 \theta_{h_1}(n_2) h_1 h_2 \]
\end{rem}

\subsection{Remarks on Automorphisms}

\begin{rem}
If $G = \langle x \rangle \simeq \ZZ_n$ and $\theta_m : G \rightarrow G$ has $x \mapsto x^m$ then $\text{Aut}(G)= \{ \theta_m : m \neq 0 \ \text{ and } \ \gcd(m,n) = 1 \}$. Thus $\text{Aut}(G) \simeq $ the group of units of $\ZZ_n$, $U(n)$. 
\end{rem}

\begin{rem}
$\ZZ_p$ has $\text{Aut}(\ZZ_p) \simeq \ZZ_{p-1}$. 
\end{rem}

\begin{rem}
If $G = \ZZ_p \times \cdots \ZZ_p$ be an elementary abelian group of rank $d$. Then $\text{Aut}(G) \simeq GL(d,p)$. \\
\\
This is because we can create an isomorphism from $G$ to a vector space $V$ of dimension $d$ over the field $\ZZ_p$, and the autmorphisms of $G$ then correspond to the linear transformations on $V$. 
\end{rem}

\begin{rem}
Let $A,B$ be groups, and $\phi : A \rightarrow B$ be a homomorphism. If $a \rightarrow b$ then $|b| \, \big| \, |a|$.  
\end{rem}



\newpage
\section{Thursday $17^{th}$ May}

\begin{eg}
Let $N = \ZZ_3 = \langle y \rangle $, $H = \ZZ_2 = \langle x \rangle$. Define the homomorphism $\theta : H \rightarrow \text{Aut}(N)$ with either $x \mapsto \theta_x = (y \mapsto y^2)$ or $x \mapsto \theta_{\text{Id}} = (y \mapsto y)$. In the second case, we see $(n_1, h_1)(n_2, h_2) = (n_1 n_2, h_1h_2)$, so the external direct product generates $\ZZ_3 \times \ZZ_2$. In the first case, $(n_1, h_1)(n_2, h_2) = (n_1 \theta_{h_1} n_2, h_1 h_2)$ which means we have $S_3$. \\
\\
We could also consider these with their presentations.
\end{eg}

\begin{eg}
Let $N = \ZZ_3 = \langle n : n^3 = 1 \rangle$ and $H = \ZZ_3 = \langle h : h^3 = 1 \rangle$.
The order of $\text{Aut}(N)$ is 2, containing the identity and inverting automorphisms. We must have $| \phi(H) | \, \big| \, |\text{Aut}(N) | = 2$, so $\theta$ must map $h$ to the identity. Thus
\[ N \rtimes H = N \times H = \langle n,h : n^3 = h^3 = 1, n^h = n \rangle. \]
\end{eg}

\begin{eg}
Let $N = \ZZ_5 = \langle n : n^5 = 1 \rangle$ and $H = \langle h : h^2 = 1 \rangle$. Then $\text{Aut}(N) = \{ \phi_1, \dots, \phi_4 \}$ where $\phi_i L: n \mapsto n^i$. We must have $h \mapsto \phi_1, \phi_4$ as $|\phi_2| = |\phi_3| = 4$. Thus
\[ N \rtimes_\theta = \langle n,h : n^5 = h^2 = 1, n^h = \theta_h(n) \rangle \]
Where $\theta : h mapsto \theta_h = \phi_1, \phi_4$. In the first case, we get the direct product $\ZZ_5 \times \ZZ_2$ and in the second we get $D_5$. 
\end{eg}

\begin{eg}
Let $|G| = pq$ where $p,q$ are distinct primes, and assume that $p > q$. Then either:
\begin{enumerate}
\item $C_p \times C_q$ (given by $C_p \rtimes_1 C_q$); or
\item if $q | p -1$, then there exists a non-trivial HM $\theta$ with $G = C_p \rtimes_\theta C_p$.
\end{enumerate}

\end{eg}


\begin{eg}
Let $N = \ZZ_5$ and $H = \ZZ_4$. Then $h$ could map to any $\phi_i$, so there are 4 possible groups. If $1_H \mapsto \phi_1$ then we have $N \times H$. If $1_H \mapsto \phi_4$ then we have $N \rtimes H \simeq D_{10}$. Otherwise, we have groups isomorphic to $\langle n,h : n^5 = h^4 = 1, n^h = \theta_i(n) \rangle$.  
\end{eg}


\begin{eg}
$D_4 \simeq \ZZ_4 \rtimes \ZZ_2$. $\text{Aut}(\ZZ_4) = \ZZ_2$. So let $\theta$ map $x \mapsto (a \mapsto a^{-1})$. Then
\[ \ZZ_4 \rtimes_\theta \ZZ_2 = \langle a, x : a^4 = x^2 = 1, a^x = a^{-1} \rangle \]
We can also see $D_4 \simeq (\ZZ_2 \times \ZZ_2) \rtimes \ZZ_2 = N \rtimes H = \langle a,b : a^2 = b^2 = [a,b] = 1 \rangle \rtimes \langle x : x^2 \rangle$. $\text{Aut}(N) = GL(2,2)$. Define $\theta: H \rightarrow GL(2,2)$ by $x \mapsto \begin{pmatrix} 1 & 0 \\ 1 & 1 \end{pmatrix} = M$. Then $M^2 = 1$. Then the external product becomes
\[ \langle a,b,x : a^2 = b^2 = x^2 = [a,b] = 1, a^x = a, b^x = ab \rangle \]
\end{eg}

\begin{rem}
A group need not be a non-trivial semidirect product, and a group can be a semi-direct product in more than one way.
\end{rem}

\begin{Ex}
Replace the 2s with some odd prime $p$ in the previous example.
\end{Ex}

\begin{eg}
Let $T = \langle a, b : a^6, b^2 = a^3 = (ab)^2 \rangle$. We will show that a non-abelian group $G$ of order 12 is isomorphic to either $A_4, D_6$ or $T$. Suppose $G \not\simeq A_4$ Let $K \in \text{Syl}_3(G)$ then $K = \langle k \rangle \triangleleft G$. Let $P \in \text{Syl}_2(G)$, so $|P| = 4$. Thus $G = KP$, and so either $P \simeq \ZZ_4$ or $P \simeq \ZZ_2 \times \ZZ_2$. In the second case, $G \simeq D_4 = S_3 \times \ZZ_2$. \\
\\
Otherwise, take $P \simeq \ZZ_4 = \langle x \rangle$, and $\ZZ_3 = \langle a \rangle$. Define $\theta : \ZZ_4 \rightarrow \text{Aut}(\ZZ_3) \simeq \ZZ_2$ by $x \mapsto (a \mapsto a^{-1})$. Consider $u = (a^2, x^2)$ and $v = (1,x)$. Then $u^6 = 1$ and $v^2 = u^3 = (uv)^2$, and $u,v \in \ZZ_3 \rtimes_\theta \ZZ_4$. Thus $G = \ZZ_3 \rtimes_\theta \ZZ_4 \simeq T$. 
\end{eg}





\newpage
\section{Monday $21^{st}$ May}

\subsection{Holomorph of $N$}
Take $N$ and $H = \text{Aut}(N)$. Set $G = N \rtimes \text{Aut}(N)$. This is called the \textbf{Holomorph} of $N$. 

\begin{eg}
Take $N = \ZZ_3$, so $H = \text{Aut}(N) \simeq \ZZ_2$. Then $|\text{Hol}(N)| = 6$ and $\text{Hol}(N) \simeq \ZZ_3 \rtimes \ZZ_2$. If $\theta$ is the identity map on $H = \{ 1, phi = (x \mapsto x^{-1}) \}$, then
\[ (n, \phi) (n^2, 11) = ( n \phi (n^2), \phi 1) = (n^2, \phi) \]
and
\[ (n^2, 1)(n, \phi) = (n^2 1(n), 1 \phi) = (1, \phi) \]
Thus the holomorph of $\ZZ_3$ does not commute, and so is isomorphic to $S_3$. 
\end{eg}


\subsection{Affine General Linear Group}

Let $F$ be a field, and $V = F^n$ be an $n$-dimensional vector space over $F$ (representing them as row vectors). Take $A \in GL(n,F)$ and $b \in F^n$. Define a map $T_{A,b} : V \rightarrow V$ by $x \mapsto xA + b$. This is called an \textbf{Affine Linear Transformation} of $V$. The set of all such transformations forms a group under composition. We call this group $AGL(n,F) = \{ T_{A,b} : A \in GL(n,F), b \in F^n \} = AGL(V)$. \\
\\
This is indeed a group, as we see that $T_{A,b} T_{C,d} = T_{AC, bC + d} \in AGL(n,F)$, and $(T_{A,b})^{-1} = T_{A^{-1}, -bA^{-1}}$. In fact, this forms a subgroup of $\text{Sym}(V)$. \\
\\
We define the normal subgroup $T(V) = \{ T_{I,b} : b \in F^n \} \triangleleft AGL(n,F)$ (i.e., the translation subgroup). Define $\phi : AGL(V) \rightarrow GL(V)$ by $T_{A,b} \mapsto A$. Then $\phi$ is an epimorphism from $AGL(V)$ onto $GL(V)$. $T(V) = \ker \phi$, and $AGL(V) = T(V) \rtimes GL(V)$. 


\begin{eg}
Let $N = \ZZ_p \times \cdots \times \ZZ_p \simeq V = GF(p)^n$. Then $GL(n,p) \simeq \text{Aut}(N)$. $AGL(n,F) = V \rtimes \text{Aut}(V) = \text{Hol}(N)$.
\end{eg}

\begin{eg}
If $n = p = 2$, then $V = \ZZ_2 \times \ZZ_2$ and $\text{Aut}(V) \simeq GL(2,2)$. $AGL(V) = V \rtimes \text{Aut}(V) \simeq (\ZZ_2 \times \ZZ_2) \rtimes S_3 = \text{Sym}(4)$. 
\end{eg}

\subsection{Wreath Products}

\begin{Def}
Take $G$, $N = G^n$, and $H \leq \text{Sym}(n)$ finite. Then the $\textbf{Permutation Wreath Product}$ is
\[ G \text{pwr} H = N \rtimes_\theta H \]
\end{Def}

The elements of $N$ are $n$-tuples of elements of $G$, $(g_1, g_2, \dots, g_n) \in N$ with each $g_i \in G$. $\theta_h$ then maps each $g_i$ to $g_{h(i)}$. We claim that $\theta_h \in \text{Aut}(N)$ (which is trivial), so $\theta : H \rightarrow \text{Aut}(N)$ is a homomorphism. $H$ acts on $N$ by permuting the $n$ direct factors. 
\begin{rem}
$|G \text{pwr} H| = |G|^n |H|$
\end{rem}


\begin{eg}
Take $G = S_3$ and $H = S_4$. Then $|G \text{pwr} H| = 6^4 \cdot 4$. 
\end{eg}


\begin{Def}
Let $H$ be a finite group of order $n$. Then the \textbf{Regular Representation} of $H$ is as a subgroup of $\text{Sym}(n)$. (note the this follows from Cayley's theorem).
\end{Def}


\begin{Def}
Take $G$, and a finite group $H \leq \text{Sym}(|H|)$. Then we define the \textbf{Regular Wreath Product} as
\[ G \text{rwr} H = ( G^n) \rtimes_\theta H \]
\end{Def}

\begin{eg}
With $G = S_3$ and $H \simeq S_4$. Then $|G \text{rwr} H | = |G|^{24} |H|$. 
\end{eg}


The regular wreath product is a subcase of the permutation wreath product. 

\begin{eg}
Let $G = \ZZ_2 = \langle x : x^2 =1 \rangle$ and $H = \ZZ_2 = \langle h = (12) \rangle$. Then $G \text{rwr} H = G \text{pwr} H = G \wreath H = W$. We see that $N = G \times G = \{ (1,1) , (1,x), (x,1), (x,x) \}$. Take $\theta : H \rightarrow \text{Aut}(N)$ with $h \mapsto \theta_h$. Then $\theta_h$ must have $(1,1) \mapsto (1,1)$, $(1,x) \mapsto (x,1)$, $(x,1) \mapsto (1,x)$ and $(x,x) \mapsto (x,x)$. \\
\\
Then $|N \rtimes H| = 2^2 \cdot 2 =8$. We see that
\[ ( (1,x), h) \cdot ((1,x). h) = ((1,x) \theta_h ((1,x)), hh) = ((1,x)(x,1), 1 ) = ((x,x),1) \neq 1 \]
Continue, to see that $|((1,x), h)| = 4$. Then set $a = ((1,x),h)$ and $b = ((1,1),h)$, so that $|b| = 2$. Also see that $a^{-1} = b^{-1} ab$. Then $(\ZZ_2 \times \ZZ_2) \rtimes \ZZ_2$ satisfies the presentation
\[ \{ a,b : a^4, b^2, a^b = a^{-1} \} \]
so $W \simeq D_4$. 
\end{eg}


\newpage
\section{Tuesday $22^{nd}$ May}

We typically call $N = G^n$ the \textbf{Base Group} of the wreath product. 

\subsection{Orders of Iterated Wreath Products}

It is clear that $| C_p \wreath C_p | = p^p \cdot p = p^{p+1}$. The $| (C_p \wreath C_p) \wreath C_p | = (p^{p+1})^p \cdot p = p^{p^2 + p + 1}$.


\begin{Lem}
Let $W_k (C_p) $ be the iterated wreath product of $C_p$ with itself $k$ times. That is, Set $W_1(C_p) = C_p$ and $W_i (C_p) = W_{i-1} (C_p)\wreath C_p$. Then 
\[ |W_k (C_p)| = p^{r(p)} \qquad \text{where} \ r(p) = p^{k-1} + p^{k-2} + \cdots + p + 1 \]
\end{Lem}

\begin{rem}
Is the wreath product commutative? Associative?
\end{rem}

\begin{Lem}
Let $p^{r(n)}$ be the highest power of the prime $p$ dividing $(p^n)!$. Then
\[ r(n) = p^{n-1} + p^{n-2} + \cdots + p + 1 = p^{n-1} + r(n-1)\]
\end{Lem}

\begin{Cor}
Let $G = \text{Sym}(p^n)$. Then $| \text{Syl}_p(G)| = p^{r(n)}$. 
\end{Cor}

\begin{eg}
If $p^n = 2^3 = 8$. Then $|\text{Syl}_2(S_8)| = 2^{2^2 + 2^1 + 1} = 2^7$.
\end{eg}

\begin{thm}
Let $p$ be a prime number, $k$ a positive integer. Then
\[ \text{Syl}_p( \text{Sym}(p^k)) = W_k (C_p) \]
\end{thm}

\begin{proof}
Induction on $k$. The base is trivial as $\text{Syl}_p(C_p) = C_p$. Then we have already observed that the highest power of $p$ dividing $p^n !$ is $p^{r(n)}$, so $|W_k(C_p)| = |\text{Syl}_p( \text{Sym}(p^k))|$. Now we want to show that $W_{k+1}(C_p)$ can be embedded as a subgroup of $\text{Sym}(p^{k+1})$, if given that $W_k$ can.  \\
\\
First, we let $N$ be the direct product of $p$ copies of  $\text{Sym}(p^k)$. These copies act on disjoint sets, so they commute. $N = \prod_p \text{Sym}(p^k) \leq \text{Sym}(p^{k+1})$. Let $H \leq \text{Sym}(p^{k+1})$ where $H = \langle h : h^p = 1 \rangle$ and $h = \prod_{i = 1}^{p^k} (i, i + p^k, \cdots, i + (p-1) p^k )$.  $h$ has $p^k$ cycles of length $p$. $|H| = p$. Thus there exists a $\theta : H \rightarrow \text{Aut}(N)$ so that $N \rtimes_\theta H \leq \text{Sym}(p^{k+1})$. \\
\\
Then we have $\text{Syl}_p(N \rtimes H) = \text{Syl}_p(N) \rtimes \text{Syl}_p(H) = \text{Syl}_p(N) \rtimes H$. But $N$ is a direct product of $\text{Sym}(p^k)$, so $\text{Syl}_p(N) = \prod \text{Syl}_p ( \text{Sym}(p^k))$. By our inductive hypothesis, this means that $N = W_k$, and $G = N \rtimes H = W_k \rtimes C_p = W_{k+1}$, so we are done.
\end{proof}

\begin{Cor}
Let $p$ be a prime. Let $n \in \NN$ be a positive integer, and take  
\[ n = a_0 + a_1 p + a_2 p^2 + \cdots + a_k p^k \]
where $0 \leq a_i \leq p-1$ be the expansion of $n$ to base $p$. Then each Sylow $p$ subgroup of $S_n$ is a direct product 
\[ T_1^{a_1} \times T_2^{a_2} \times \cdots \times T_k^{a_k} \]
where $T_i$ is a Sylow $p$-subgroup of $\text{Sym}(p^i)$, i.e., $T_i = W_i (C_p) = \text{Syl}_p ( \text{Sym}(p^i))$. 
\end{Cor}

\begin{proof}
Calculate the order of $p$ dividing $n!$. The number of integers between $1$ and $n$ divisible by $p$ is $[n / p ]$. The number of these divisible by $p^2$ is $[n / p^2]$. Therefor $p^m | n$ where $m = [n / p] + [n / p^2] + \cdots$. So
\begin{align*}
 m     &= (a_1 p + a_2 p^2 + \cdots a_k p^k) + (a_2 p_2 + \cdots + a_k p^k) + \cdots + (a_k p^k) \\
       &= a_1 + a_2 ( p + 1) + a_3 ( p^2 + p + 1) + \cdots \\
\end{align*}
Thus $p^m$ is the order of $T_1^{a_1} \times \cdots \times T_k^{a_k}$. Then, as in the proof of the theorem, divide the set of size $n$ into disjoint subsets, with each corresponding to some $a_i$ with size $p^i$. The symetric groups of these sets are then $W_i(C_p)$. We can then argue as in the theorem to see that this is isomorphic to  subgroup of $S_n$. 
\end{proof}


\begin{eg}
Let $G = \text{Sym}(5)$. $5 = 1 \cdot 2^0 + 0 \cdot 2 + 1 \cdot 2^2$. Then $\text{Syl}_2(G) = T_{4} = \text{Syl}( \text{Sym} (4) ) = C_2 \wreath C_2 \simeq D_4$.  
\end{eg}





\newpage
\section{Thursday $24^{th}$ May}

\begin{eg}
Taek $p^3 = 27$. Then $N = \text{Sym}(9) \times \text{Sym}(9) \times \text{Sym}(9)$ and $H = C_3$. Then $G = N \rtimes H \leq S_{27}$. \\
\\
Then $\text{Syl}_3(G) = \text{Syl}_3(N) \rtimes H = \prod \text{Syl}_3(S_9) \rtimes H$. This equals
\[ (C_3 \wreath C_3) \times (C_3 \wreath C_3) \times (C_3 \wreath C_3)) \rtimes C_3 = (C_3 \wreath C_3) \wreath C_3 \] 
\end{eg}

\begin{eg}
Take $G = S_{10}$. $10 = 0 + 1 \times 2 + 1 \times 2^3$. The $2$ terms corresponds to $S_2$, so $T_2 = \ZZ_2$. The $2^3$ term corresponds to $S_{2^3}$, with Sylow 2-subgroup $(\ZZ_2 \wreath \ZZ_2)$. Thus
\[ \text{Syl}_2(G) =  \ZZ_2 \times ((\ZZ_2 \wreath \ZZ_2) \wreath \ZZ_2) \]
Similarly, $10 = 1 + 3^2$ so $\text{Syl}_3(G) = \ZZ_3 \wreath \ZZ_3$. Lastly, $10 = 2 \times 5$ so $\text{Syl}_5(G) = \ZZ_5 \times \ZZ_5$. 
\end{eg}

\begin{eg}
Let $G = S_{34}$. $34 = 1 + 2 \cdot 3 + 3^3$. Thus
\[ \text{Syl}_3(G) = T_{3^1}^2 \times T_{3^3}^1 = (\ZZ_3 \times \ZZ_3) \times (( \ZZ_3 \wreath \ZZ_3) \wreath \ZZ_3) \]
\end{eg}

\subsection{General Construction of Wreath Products}

Take $G \leq \text{Sym}(\Lambda)$, $H \leq \text{Sym}(\Omega)$, and set $W = G \wreath H \leq \text{Sym}( \Lambda \times \Omega)$. Given some $g \in G$ and $w \in \Omega$, define a permutation of $g^*_w$ of $\Lambda \times \Omega$ as follows: for $(\lambda, \omega) \in \Lambda \times \Omega$ define 
\[ g^*_w (\lambda, \omega) = \begin{cases} (\lambda^g, \omega) &\text{if } \omega = w \\ (\lambda, \omega) &\text{Otherwise} \end{cases} \]
From this, note that $g_w^* (g')^*_w = (gg')^*_w$ for all $g, g' \in G$. Define $G^*_w = \{ g^*_w : g \in G \} \leq \text{Sym}(\Lambda \times \Omega)$. The the map $G \rightarrow G^*_w$ taking $g$ to $g^*_w$ is a isomorphism for all $w \in \Omega$. Also define $h^*$ as the permutation of $\Lambda \times \Omega$ taking $(\lambda, \omega)$ to $(\lambda, \omega^h)$, and $H^* = \{ h^* :  h \in H \}$. Similarly, $h \mapsto h^*$ is an isomorphism.

\begin{thm}
Given $G \leq \text{Sym}(\Lambda)$, and $H \leq \text{Sym}(\Omega)$ (where $\Omega$ is finite), the wreath product
\[ G \wreath H \simeq W = \langle G^*_w, H^* : w \in \Omega \rangle \leq \text{Sym}(\Lambda \times \Omega) \]
\end{thm}
\begin{proof}
Define $K^* = \langle \cup_{w \in \Omega} G^*_w \rangle$. We claim that this is equal to the direct product
\[ K^* = \prod_{w \in \Omega} G^*_w \]
note that $G^*_w$ centralises $G^*_\omega$ for all $\omega \neq w$. Thus $G^*_w \triangleleft K^*$, and $G^*_w \cap \langle \cup_{\omega \in \Omega \setminus \{ w \}} G^*_\omega \rangle = 1$, proving the claim. Now take $h \in H$ and $w \in \Omega$. We have $h^* g^*_w (h^*)^{-1} = g^*_{w^h} \in K^*$, thus $h^* K^* (h^*)^{-1} \leq K^*$ for all $h \in H$, and thus $K^* \triangleleft W = \langle K^*, H^* \rangle$. Therefor $W = K^* H^*$. \\
\\
Take some element $x \in K^* \cap H^*$. This must fix the point $(\lambda, \omega)$, so it must be the identity. Thus
\[ W = \langle G^*_w, H^* : w \in \Omega \rangle = \langle K^*, H^* \rangle = K^* \rtimes H^* = \left( \prod_{\omega \in \Omega} G^*_\omega \right) \rtimes H^*\]
Then the map $\phi : G \wreath H \rightarrow W$ given by $(g_\omega)_{\omega \in \Omega}h \mapsto (g^*_\omega)h^*$ is an isomorphism.
\end{proof}

\begin{eg}
Take $G = H = C_2$, set $\Lambda = \{ 1, 2 \}$ and $\Omega = \{ 3, 4 \}$. Then label the elements of $\Lambda \times \Omega$ as
\[ \{ a = (1,3), b = (2, 3), c = (1, 4), d = (2,4) \} \]
Set $g = (12)$, so then the permutation $g^*_3$ has
\begin{align*}
a = (1,3) &\mapsto (1^g, 3) = (2,3) = b \\
b = (2,3) &\mapsto (2^g, 3) = (1,3) = a\\
c = (1,4) &\mapsto (1, 4) = c \\
d = (2,4) &\mapsto (2, 4) = d
\end{align*}
or equivalently, $g^*_3 = (ab)$. Similarly $g^*_4 = (cd)$. Lastly, $h^*$ fixes the first component and permutes the second, so $h^* = (ac)(bd)$. So we have
\[ W = (\langle (ab) \rangle \times \langle (cd) \rangle ) \rtimes \langle (ac)(bd) \rangle = \langle (ab), (cd), (ac), (bd) \rangle \leq \text{Sym}(\{ a,b,c,d\}) \]
Computing this out show that $W \simeq D_4$.
\end{eg}


\newpage
\section{Monday $28^{th}$ May}

\begin{eg}
Take $G = C_2 = \langle (12) \rangle$ and $H = C_3 = \langle (345) \rangle$, $\Lambda = \{ 1,2 \}$ and $\Omega = \{ 3, 4, 5\}$. Label
\[ \Lambda \times \Omega = \{ a = (1,3), b = (2,3), c = (1,4), d = (2,4), e = (1,5), f = (2,5) \} \]
Take $\omega \in \Omega$. Then $g_\omega^*$ has $g_3^* = (ab)$, $g_4^* = (cd)$ and $g_5^* = (ef)$. If $h = (345)$, then $h^* = (ace)(bdf)$. Thus $W = G \wreath H = \langle (ab), (cd), (ef), (ace)(bdf) \rangle  \leq S_6$, and $|W| = |G|^n |H| = 2^3 \cdot 3$.
\end{eg}

\begin{eg}
Consider $W = C_3 \wreath C_2 = \langle (123) \rangle \wreath \langle (45) \rangle$. Then 
\[ \Lambda \times \Omega = \{ 1,2,3 \} \times \{4,5\} = \{ a = (1,4), b = (2,4), c = (3,4), d = (1,5), e = (2,5), f = (3,5) \} \]
$g = (123)$ so $g_4^* = (abc)$ and $g_5^* = (def)$. $h = (45)$ so $h^* = (ad)(be)(cf)$.  Thus
\[ W = \langle (abc), (def), (ad)(be)(cf) \rangle \]
with order $3^2 \cdot 2$.
\end{eg}

\begin{eg}
Take $P(n)$ to be the group of permutation matrices. So $P(n) \simeq S_n$. e.g., for $n = 3$, 
\[ (123) \sim \begin{pmatrix} 0 & 1 & 0 \\ 0 & 0 & 1 \\ 1 & 0 & 0 \end{pmatrix} \]
Also define the group $B(n)$ as the group generated by permutation matrices where single non-zero entry can be $\pm 1$. This is sometimes called the Monomial group or the Hyper octahedral group. Observe that $B(n)$ has a normal subgroup $D$ consisting of diagonal matrices with $\pm 1$ on the diagonal. Each matrix in $B(n)$ can be written as a product of a diagonal matrix and a permutation matrix. Thus $B(n) = D \rtimes P(n)$. Observe that $D \simeq (Z_2 )^n$, so $B(n) = \ZZ_2 \wreath S_n$, and so $|B(n)| = 2^n n!$.
\end{eg}

\begin{rem}
Recall the Pr\"{u}fer group $\ZZ_p^\infty = \{ z \in \CC : z^{p^k} = 1 \text{ for some } k \in \NN \}$. Note that this is abelian because $\CC$ is. The only proper subgroups are cyclic of the form $T_n = \{ z \in \ZZ_p^\infty : z^{p^n} = 1 \}$. There is no maximal subgroup. The group $W = \ZZ_p \wreath \ZZ_p^\infty$ is thus an infinite group with trivial center, and is not nilpotent. 
\end{rem}


\newpage
\section{Galois Theory - Just for fun}

Take fields $\FF \leq k$, with $k$ algebraically closed. Take $f(x) \in \FF[x]$, so then $f(x) = a (x - \alpha_1) \cdots (x - \alpha_n)$ for $a \in \FF$ and $\alpha_i \in k$. $\FF (\alpha)$ is the smallest subfield of $k$ containing $\alpha$. Call $\FF( \alpha_1, \dots, \alpha_n)$ the splitting field of $f(x)$ over $\FF$. 

\begin{eg}
take $f(x) = x^2 + 1 \in \QQ[x] \subseteq \RR[x]$. Then the splitting field for $f$ over $\QQ$ is $\QQ(i)$. The splitting field for $f$ over $\RR$ is $\RR(i) = \CC$. 
\end{eg}

\begin{Def}
Call $\EE$ and extension field of $\FF$ if $\EE \subseteq \FF$ and operations of $\FF$ are those of $\EE$ restricted to $\FF$. 
\end{Def}

\begin{rem}
$\EE$ is a vector space over $\FF$, when the elements are viewed as vectors, with scalars $\lambda \in \FF$. Call $d = \dim E$ the degree of the extension $[\EE : \FF]$. 
\end{rem}

Take $p(x) \in \FF[x]$ an irreducible polynomial of degree $n$. Then
\[ \EE[x] = \frac{\FF[x]}{\langle p(x) \rangle} \]

\begin{eg}
$[\CC : \RR] = 2$ with basis $\{ 1, i\}$. $\CC$ is an infinite degree extension of $\QQ$. 
\end{eg}


\begin{Def}
An automorphism of $E$ is a ring isomorphism on itself. We say $\sigma \in \text{Aut}(\EE)$ fixes $\FF$ point-wise if $\sigma(c) = c$ for all $c \in \FF$. 
\end{Def}


Call $\sigma : x \mapsto x^p$ the Frobenius map. 


\begin{Lem}
Take extension field $\EE$ of $\FF$, $f(x) \in \FF[x]$, and $\sigma \in \text{Aut}(\EE)$ which fixes $\FF$ pointwise. If $\alpha \in \EE$ is a root of $f(x)$, then $\sigma(\alpha)$ is also a root. 
\end{Lem}

\begin{proof}
Expand $f(x) = \sum_{i = 0}^n c_i x^i$ with $c_i \in \FF$, so that $f(\alpha) = \sum_{i=0}^n c_i \alpha^i = 0$. But then $\sigma(c_i) = c_i \in \FF$, so
\[ 0 = \sigma( f(\alpha) ) = \sum_{i = 1}^n \sigma( c_i \alpha^i ) = \sum_{i = 0}^n c_i \alpha^i \]
so $\sigma(\alpha)$ is also a root.
\end{proof}

\begin{Def}
If $\EE$ extends $\FF$ and is a splitting field of $f(x) \in \FF[x]$, then define
\[ \text{Gal}(E | F) = \{ \sigma \in \text{Aut}(E) : \sigma \text{ fixes } \FF \text{ pointwise } \} \]
then $\text{Gal}(E | F)$ is a group under composition.
\end{Def}

\begin{thm}
If $f(x) \in \FF[x]$ has $n$ distinct roots in  the splitting field $n$, then $\text{Gal}(\EE : \FF) \simeq H \leq S_n$. 
\end{thm}

\begin{proof}
Take the roots $X = \{ \alpha_1, \dots, \alpha_n \}$. Take $\sigma \in \text{Gal}(\EE | \FF)$, so then $\sigma(X) = X$. Then $\theta : \text{Gal}(\EE | \FF) \rightarrow S_n$ given by $\sigma \mapsto \sigma |_X$ is a homomorphism, and so we are done.
\end{proof}


\begin{Def}
Every $f(x) \in \FF[x]$ has an irreducible factorisation:
\[ f(x) = a p_1(x) \cdots p_r(x) \]
Call $f(x)$ separable if each $p_i$ has no repeated roots. 
\end{Def}

\begin{thm}
If $f(x) \in \FF[x]$ is seperable in  $\EE$, then $|\text{Gal}(\EE | \FF) = [\EE : \FF]$.
\end{thm}

\begin{eg}
$\QQ(\sqrt{2}) = \{ a + b \sqrt{2} : a,b \in \QQ \}$. Let $\phi \in \text{Gal}(\QQ(\sqrt{2}) | \QQ)$, then $2 = \phi(2) = \phi( \sqrt{2} \sqrt{2}) = (\phi(\sqrt{2}))^2$. Thus $\phi(2) = \pm \sqrt{2}$. Thus $\text{Gal}( \QQ(\sqrt{2}) | \QQ)$ has just two elements, the identity, and the conjugation / rotation map $a + b \sqrt{2} \mapsto a - b \sqrt{2}$.  
\end{eg}

\begin{eg}
Take $f(x) \in \RR[x]$, which splits over $\CC$. Then $|\text{Gal}( \CC | \RR)| \leq 2$.If $\sigma$ is a non trivial element of this Galois group, then it is again complex conjugation: $z = a + ib \mapsto a - bi = \overline{z}$. 
\end{eg}

\begin{thm}[Fundamental theorem of Galois Theory]
If $\FF$ is a finite field or has characteristic $0$, and $\EE$ is a splitting field over $\FF$ of some $f(x) \in \FF[x]$, then the mapping from the set of subfields of $k$ of $\EE$ which contain $\FF$, to the set of subgroups of $\text{Gal}(\EE | \FF)$ given by $k \mapsto \text{Gal}(\EE | k)$ is injective. 
\end{thm}

\begin{proof}
If $\FF \leq k \leq \EE$, then we have:
\begin{enumerate}
\item $|\EE : k| = | \text{Gal}( \EE : k )|$ and $[k : \FF] = | \text{Gal}(\EE | \FF) | / |\text{Gal}(\EE |k )|$; and
\item If $k$ is a splitting field of $f(x) \in \FF[x]$, then $\text{Gal}(\EE | k ) \triangleleft \text{Gal}( \EE | \FF )$ and $\text{Gal}(k | \FF) \simeq \text{Gal}(\EE | \FF) / \text{Gal}(\EE | k)$. 
\end{enumerate}

\end{proof}



\begin{Def}
Take $f(x) \in \FF[x]$, and $\EE$ a splitting field. Then $f(x)$ is solvable by radicals if there exists a chain of subfields $(k_i)$ satisfying $E \subseteq k_t$ for some $t$ and each $k_{i+1}$ is obtained by adjoining a root of an element of $k_i$.   \\
\\
That is, $k_{i+1} = k_i (\beta_{i+1})$ so that $\beta_{i+1} \in k_{i+1}$, and $\beta_{i+1}^n \in k_i$. 
\end{Def}

\begin{eg}
take $f(x) = ax^2 + bx + c$ and $g(x) = ax^2 + c - \frac{1}{4} b^2 = f( x - \frac{1}{2} b)$. ($g$ is called the reduced polynomial of $f$. Thus $\mu(x) = \pm \sqrt{b^2 - 4ac}$ is a root of $g(x)$ if and only if $\frac{-b \pm \mu(x)}{2a}$ is a root of $f$. \\
\\
Let $\FF = \QQ (b ,c)$ so $f(x) \in \FF[x]$. If $\beta = \sqrt{b^2 - 4ac}$ then $\beta^2 \in \FF$. So $k_1 = \FF(\beta)$ and $k_0 = \FF$.
\end{eg}

\begin{eg}
Take $f(x) = x^3 + ax^2 + bx + c$, with reduced form $f( x - \frac{1}{3} a) = g(x) = x^3 + qx + r$. Again $\alpha$ is a root of $g$ iff $\alpha - \frac{1}{3} a$ is a root of $f$. The roots of $g$ are
\[ y + \zeta, \qquad wy + w^2 \zeta, \qquad w^2 y + w \zeta \]
where $w$ is a cube root of unity and:
\begin{enumerate}
\item $R = r^2 + \frac{4}{27} q^3$ 
\item $y^3 = \frac{1}{2}( - r + \sqrt{R})$
\item $\zeta = \frac{-q}{3y}$
\end{enumerate}

Then the chain of field extensions are 
\begin{align*}
k_0 &= \FF = \QQ ( q, r) \\
k_1 &= k_0 ( \sqrt{ R }) \\
k_2 &= k_1 ( {}^3\sqrt{-r + \sqrt{R}}) \\
k_3 &= k_2 (w)
\end{align*}
 
\end{eg}

\begin{thm}[Abel and Rufini's Theorem]
If $f(x)$ is solvable by radicals, then every root of $f(x)$ has some expression in the coefficients of $f(x)$ involving field operations and extraction of roots. 
\end{thm}

\begin{thm}[Galois' Theorem]
Let $f(x) \in \FF[x]$ be solvable by radicals over a field $\FF$ of character $0$. Let $\EE$ be the corresponding splitting field. Then $\text{Gal}(\EE | \FF)$ is soluble.
\end{thm}


\begin{eg}
The exist quintic polynomials $f(x) = \QQ[x]$ that are not solvable by radicals. Consider $f(x) = x^5 - 4x + 2 \in \QQ[x]$. It is irreducible by Eisenstein's criterion. Take $G = \text{Gal}( \EE : \QQ)$. $f(x)$ has three real roots and two complex conjugates. $G \simeq H \leq S_5$. We must have $5 | |G|$, so by Cauchy $G$ has a five cycle. $G$ can only permute the complex roots, so $G$ has a two cycle. Thus $H = S_5$ and so is insoluble. 
\end{eg}


\begin{rem}
Every finite group is a Galois group over some field. The following remains unsolved:
\begin{center}
Which finite groups can arise as Galois groups over $\QQ$.
\end{center}
It is conjectured that all can. 
\end{rem}
































 




\end{document}                
